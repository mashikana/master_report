\chapter*{Abstract}
\addcontentsline{toc}{chapter}{Abstract}

\par
Conventional single rotor helicopters with $N$ blades generally experience 1/rev and N /rev vibration.
Regarding $N$/rev vibration, for the case of $N=4$, it was confirmed from the analysis using
CAMRAD $\mathrm{II}$ (Comprehensive Analytical Model of Rotorcraft Aerodynamics and Dynamics),
that it is possible to alleviate high angle of attack, which a blade experiences in the azimuthal region
on the retreating side by changing the trim, and it is possible to do so by changing the CG position toward leftward and also rearward,
when a main rotor rotates in a CW direction.
Flight demonstration tests have been conducted for a 4 bladed radio-controlled helicopter with a rotor diameter of about 1.5 m with its CG position varied.
Flight tests have been conducted for four typical CG positions, and the test results are what were expected to be and show that the vibration can be reduced.
This research outcome brings pilot fatigue reduction, reduces the failure rate of equipment of helicopters, and can improve accuracy of atmospheric observation using helicopters. 