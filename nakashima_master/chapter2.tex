\chapter{トリム解析}
\begin{flushright}
	\begin{minipage}{0.8\hsize}
		\quad 本章では,可変ピッチロータ搭載型のマルチロータ機に関して,遠心力によるねじり下げ発生の要因と解析対象の機体について示す.
		まず,課題となっている遠心力によるねじり下げモーメントについて機体モデルを設定する.
        次に,ねじり下げモーメント導出の際の手順について示す.
        最後に解析対象のブレードのモデルについて示す.
	\end{minipage}
\end{flushright}
    \section{CAMRAD $\mathrm{II}$ を用いたトリム解析}
    \par
    第1章で述べたように回転翼機の振動はほとんどがメインロータの空力荷重に由来するものである.
    この振動荷重およびモーメントを解析的に精度よく求めるためにCAMRAD I$\mathrm{II}$を使用する.
    本論文では水平飛行での周期的な振動を議論の対象としているため,水平飛行状態で機体に作用するメインロータ,テールロータ,水平尾翼,
    垂直尾翼,胴体空力荷重および重力がバランスする状態をトリム解析で求める.
    トリム解析のなかでメインロータハブ中央位置に作用する振動荷重およびモーメントも併せて得られる.
    2.2項では重心移動を実施した場合に得られるトリム変数が物理的に正しいものかを考察し,解析が正しく実行されていることを確認する.
    ここで メインロータ回転座標系におけるn/rev荷重(n:整数)を正しく見積もるためにメインロータブレードは弾性体として取り扱われる必要があり,
    かつ振動荷重の精度向上のためにブレードのチップボルテックスの取り扱いには専用のオプションを導入した.
    図2-1に CAMRAD $\mathrm{II}$のトリム解析の流れを示す.
    大きくはメインロータブレード,テールロータブレードの弾性体としての変形を含むフラッピング,リードラグ,フェザリング応答とインフロー計算部での収束計算,
    また全機の6自由度の荷重およびモーメントのバランスにおける収束計算から成り立っており,この計算の早期収束性を高めるためにインフローや機体姿勢角などの初期値の設定が重要となってくる.
    インフロー計算に関しては Free Wake Modelでの収束計算については解析結果を予測しがたいため,
    LEVEL 1 (Momentum Theory)→ LEVEL 2 (Prescribed Wake Model)→ LEVEL 3 (Free Wake Model)の順番に段階を上げて収束計算を図っている.
    初期条件としては Drees Modelを定義しており 式 (2-1)~式 (2-4)に示されるように定義される.

    \begin{equation}
        \zeta = \zeta_{0} + \zeta_{1c}\frac{r}{R}\cos\Psi +  \zeta_{1s}\frac{r}{R}\sin\Psi
    \end{equation}
    
    ここで,

    \begin{equation}
        \zeta_{0} = \frac{c_{T}}{2\sqrt{\lambda^2_{0}+ \mu^2}} +  \mu\tan(-\Theta)
    \end{equation}
    
    \begin{equation}
        \zeta_{1c} = \frac{c_{T}}{2\sqrt{\lambda^2_{0}+ \mu^2}}\frac{4}{3}[(1-8\mu^2)\sqrt{1+(\frac{\lambda}{\mu})^2} - \frac{\lambda}{\mu}]
    \end{equation}

    \begin{equation}
        \zeta_{1s} = \frac{c_{T}}{2\sqrt{\lambda^2_{0}+ \mu^2}}(-2\mu)
    \end{equation}

    インフローの収束が得られたのちに,胴体に作用する各コンポーネント(メインロータ,テールロータ,胴体,水平/垂直尾翼)からの空力荷重および
    重力が重心位置でバランスした状態での機体座標系における荷重およびモーメント,計6自由度の数式(後述する式 (5-7)参照)における左辺項を
    $\bar{F}=\left \lfloor X, \quad Y, \quad Z, \quad L, \quad M, \quad N \right \rfloor^T$と 表記してトリム変数である
    $\eta=\left\lfloor\Theta, \quad \Psi, \quad \theta_0, \quad \theta_{1 c}, \quad \theta_{1 s}, \quad \theta_{t r}\right\rfloor^T$について
    初期値を$\eta_{0}$としたときに$\bar{F}$を0とするためにTaylorの1次展開式を用いて記載すると

    \begin{equation}
        \bar{\bar{F}}\left(\eta_0+\Delta \eta\right)=\bar{F}\left(\eta_0\right)+\left.\Delta \eta \frac{\partial \bar{F}(\eta)}{\partial \eta}\right|_{\eta=\eta_0}=0
    \end{equation}

    となり,トリム収束解を得るための次のステップとして$\Delta\eta$は\eqref{deltaeta}式の様に表現できる.
 
    \begin{equation}
        \Delta \eta=\left[\left.\frac{\partial \bar{F}(\eta)}{\partial \eta}\right|_{\eta=\eta_0}\right]^{-1} \bar{F}\left(\eta_0\right) \label{deltaeta}
    \end{equation}
 

    ここで,$\left.\frac{\partial \bar{F}(\eta)}{\partial \eta}\right|_{\eta=\eta_0}$は$\bar{F}(\eta)$の
    $\eta=\eta_0$におけるヤコビ行列であり\eqref{jacobian}に示す行列となる

   \begin{equation}
            \left.\frac{\partial \bar{F}(\eta)}{\partial \eta}\right|_{\eta=\eta_0}=\left[\begin{array}{cccccc}
            \frac{\partial X}{\partial \alpha_s} & \frac{\partial X}{\partial \varphi_s} & \frac{\partial X}{\partial \theta_0} & \frac{\partial X}{\partial \theta_{1 c}} & \frac{\partial X}{\partial \theta_{1 s}} & \frac{\partial X}{\partial \theta_{t r}} \\
            \frac{\partial Y}{\partial \alpha_s} & \frac{\partial Y}{\partial \varphi_s} & \frac{\partial Y}{\partial \theta_0} & \frac{\partial Y}{\partial \theta_{1 c}} & \frac{\partial Y}{\partial \theta_{1 s}} & \frac{\partial Y}{\partial \theta_{t r}} \\
            \frac{\partial Z}{\partial \alpha_s} & \frac{\partial Z}{\partial \varphi_s} & \frac{\partial Z}{\partial \theta_0} & \frac{\partial Z}{\partial \theta_{1 c}} & \frac{\partial Z}{\partial \theta_{1 s}} & \frac{\partial Z}{\partial \theta_{t r}} \\
            \frac{\partial L}{\partial \alpha_s} & \frac{\partial L}{\partial \varphi_s} & \frac{\partial L}{\partial \theta_0} & \frac{\partial L}{\partial \theta_{1 c}} & \frac{\partial L}{\partial \theta_{1 s}} & \frac{\partial L}{\partial \theta_{t r}} \\
            \frac{\partial M}{\partial \alpha_s} & \frac{\partial M}{\partial \varphi_s} & \frac{\partial M}{\partial \theta_0} & \frac{\partial M}{\partial \theta_{1 c}} & \frac{\partial M}{\partial \theta_{1 s}} & \frac{\partial M}{\partial \theta_{t r}} \\
            \frac{\partial N}{\partial \alpha_s} & \frac{\partial N}{\partial \varphi_s} & \frac{\partial N}{\partial \theta_0} & \frac{\partial N}{\partial \theta_{1 c}} & \frac{\partial N}{\partial \theta_{1 s}} & \frac{\partial N}{\partial \theta_{t r}}
            \end{array}\right]_{\eta=\eta_0}    \label{jacobian}
    \end{equation}

    \eqref{deltaeta}式で得られる$\Delta\eta$を用いて初期値$\eta_{0}$に$K\Delta\eta(0 < K < 1)$を加算して図2-1に示されるフロー全体が収束するまで,
    つまりトリム変数$\eta$が収束するまで計算を実施することでトリム計算が完了することになる.

    \section{トリム解析結果}
    \par

    本論文では解析の対象となる機体はラジコンヘリコプタとした.(表2-1)


    \par
    数値解析はラジコンヘリコプタが30.8 m/s(120 kt)で高速巡航飛行している状態でのトリム解析を実施した.
    特に振動を正確に計算する目的からメインロータブレードは弾性体としてモデル化され,20分割されている.
    各々の要素は断面特性を正確に計算されたモデルである.
    ここで 胴体に関しては剛体としてモデル化している.
    これは詳しくは後述するが,メインロータシャフト,及びメインギアボックスが取りつく構造部位が有意に変形するような胴体の振動モードは
    4/rev(25.0 Hz)からは遠く離れておりメインロータと胴体の連成を考慮に入れなくても問題は生じないためである.
    CAMRAD $\mathrm{II}$によるトリム解析は図 2-2に示すようにベースラインCGから前方に 0.305 m,後方に1.68 mの範囲で,
    また左/右舷 0.912 mの範囲でそれぞれ 0.152 mの刻み幅で設定された4箇所で実施した.
    機体の慣性モーメントはどのCGの位置でもベースラインの値に統一して一定の値を使用した.
    この仮定の下においても胴体の4/rev振動の評価には影響がないから問題はない.
    この詳細についても後述する.

    \clearpage 

    \par
    その他の条件については例えばメインロータブレードの弾性変形応答に関わる構造減衰率や自由後流モデルでインフローに影響を与える
    Wakeの周期数などあるが,ここではベースラインで使用している値と同一のものを全ての評価点で使用した.
    また,特に水平尾翼の取付角はベースラインのラジコンヘリコプタと同じく 0 degとした.
    図2-3にトリム変数の結果(ピッチ,ロール,フラッピング角)を示す.
    また,図 2-4には 図 2-2に示される胴体及び水平尾翼の空力作用点に作用する機体座標系における$F_x$及び$F_z$を示す.
    それぞれのグラフの横軸は図 2-1に示される$y$軸を示し,また縦軸は$x$軸を表しており,重心位置が変化した際のそれぞれの値のコンター図となっている.
    それぞれの結果から次に示す所見が得られる.


    \section{ロータブレード}

    \par 
    Table.にロータのピッチタイプを示す.ロータの種類については固定ピッチロータと可変ピッチロータがある.
    固定ピッチロータは低コスト,軽量で扱いやすいのが利点としてあげられる.しかしながら,角速度の変化で制御を行うため,突風のような外乱による姿勢変動に
    対しての応答が遅いという欠点がある.それに対して,可変ピッチロータはピッチ角の増加によって制御を行うことで,応答を早くすることが可能である.
    可変ピッチロータはピッチを変化させるためのサーボモータが必要であり,搭載するサーボモータの規格に応じて,応答の速さは決まる.
    本研究ではアスペクト比,ソリディティの二つの観点に着目して検討を行う.Table.\ref{table:ar},Table.\ref{table:solidity}にそれぞれの特徴を示す.高アスペクト比のブレードは揚抗比が大きく,空力性能が高い.
    しかし,翼端失速が発生する限界ピッチ角が小さく,可変ピッチロータ搭載型マルチロータ機には適していない.また,空気抵抗が大きいという欠点もある.
    低アスペクト比のブレードは揚抗比は高アスペクト比のものに劣るが,空気抵抗も小さく,翼端失速が発生する限界ピッチ角も大きい.
    ソリディティについては,低いと重量,空気抵抗が小さく,構造も単純である.ソリディティが高いと揚力,推力が大きくなり,空力性能は向上する.
    
     
    

    \begin{table}[H]
        \centering
        \caption{Aspect ratio}
        \label{table:ar}
        \setlength{\tabcolsep}{5pt}
        \begin{tabular}{|c|c|c|} \hline
            Aspect ratio &   Low &   High \\ \hline
            merit &   
            \begin{tabular}[c]{@{}c@{}}
                large critical pitch angle  \\
                low aerodynamic drag
            \end{tabular}
            &
            \begin{tabular}[c]{@{}c@{}}
                high lift-drag ratio
            \end{tabular}
            \\ \hline
            demerit & 
            \begin{tabular}[c]{@{}c@{}}
                low lift-drag ratio
            \end{tabular}
            & 
            \begin{tabular}[c]{@{}c@{}}
                small critical pitch angle  \\
                high aerodynamic drag
            \end{tabular}
            \\ \hline
        \end{tabular}
    \end{table}

    \begin{table}[H]
        \centering
        \caption{Solidity}
        \label{table:solidity}
        \setlength{\tabcolsep}{5pt}
        \begin{tabular}{|c|c|c|} \hline
            Solidity    &   Low & High \\ \hline
            merit &   
            \begin{tabular}[c]{@{}c@{}}
                light weight, low aerodynamic drag \\
                simple structure
            \end{tabular}
            &
            \begin{tabular}[c]{@{}c@{}}
                large lift and thrust
            \end{tabular}
            \\ \hline
            demerit & 
            \begin{tabular}[c]{@{}c@{}}
                small lift and thrust
            \end{tabular}
            & 
            \begin{tabular}[c]{@{}c@{}}
                heavy weight, high aerodynamic drag \\
                complex structure
            \end{tabular}
            \\ \hline
        \end{tabular}
    \end{table}
    

    \clearpage

    \section{ブレード解析モデル}
    \par
    Fig.に解析対象となるブレードモデルを示す.ブレードの直径は$0.686\ \mathrm{m}$,平面形はテーパ比0.603のテーパ翼で統一した.
    ブレードの翼型はOAF117で検討する.ブレードの表皮はすべて$3\ \mathrm{mm}$であるとする.Table.のようなブレード諸元について
    それぞれのパターンでねじり下げモーメントを計算する.2種類の翼弦長,翼枚数2枚または4枚の組み合わせで比較を行う.ブレード全体にはたらくねじり下げモーメントは21度の
    線形のねじり下げがあるものとする.WB2は翼弦長が同じであり,他の2ブレードに対して翼弦長が大きく,低アスペクト比のブレードが用いられており,
    NB2,NB4の翼弦長(アスペクト比)は同一で,WB2の翼弦長の半分である.翼枚数を考慮したブレードソリディティは
    WB2,NB4で同一の値となり,これらに対してNB2のソリディティは半分である.
    
<<<<<<< HEAD
=======

    
>>>>>>> 66e3e54c0511da761b21ec75822640018e35ff0d

    

    