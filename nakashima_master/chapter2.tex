\chapter{先行研究}
\begin{flushright}
	\begin{minipage}{0.8\hsize}
		\quad 本章では,先行研究としてシングルロータヘリコプタSH-60Kに関して,トリム解析とを実施した結果と振動評価指標について示す.
		まず,トリム解析の手順について示す.
        次に,検討対象であるシングルロータヘリコプタについて機体モデルを設定する.
        そして,トリム解析によって得られた結果について示す.
        さらに,
	\end{minipage}
\end{flushright}
    \section{CAMRAD $\mathrm{II}$ を用いたトリム解析}
    \par
    第\ref{chap:intro}章で述べたように回転翼機の振動はほとんどがメインロータの空力荷重に由来するものである.
    この振動荷重およびモーメントを解析的に精度よく求めるためにCAMRAD $\mathrm{II}$を使用する.
    本論文では水平飛行での周期的な振動を議論の対象としているため,水平飛行状態で機体に作用するメインロータ,テールロータ,水平尾翼,
    垂直尾翼,胴体空力荷重および重力がバランスする状態をトリム解析で求める.
    トリム解析のなかでメインロータハブ中央位置に作用する振動荷重およびモーメントも併せて得られる.
    2.2項では重心移動を実施した場合に得られるトリム変数が物理的に正しいものかを考察し,解析が正しく実行されていることを確認する.
    ここで メインロータ回転座標系におけるn/rev荷重(n:整数)を正しく見積もるためにメインロータブレードは弾性体として取り扱われる必要があり,
    かつ振動荷重の精度向上のためにブレードのチップボルテックスの取り扱いには専用のオプションを導入した.
    Fig.\ref{fig:trimanlysis_flow}にCAMRAD $\mathrm{II}$のトリム解析の流れを示す.
    大きくはメインロータブレード,テールロータブレードの弾性体としての変形を含むフラッピング,リードラグ,フェザリング応答とインフロー計算部での収束計算,
    また全機の6自由度の荷重およびモーメントのバランスにおける収束計算から成り立っており,この計算の早期収束性を高めるためにインフローや機体姿勢角などの初期値の設定が重要となってくる.
    インフロー計算に関しては Free Wake Modelでの収束計算については解析結果を予測しがたいため,
    LEVEL 1 (Momentum Theory)→ LEVEL 2 (Prescribed Wake Model)→ LEVEL 3 (Free Wake Model)の順番に段階を上げて収束計算を図っている.
    初期条件としては Drees Modelを定義しており,Eq.\ref{zeta},Eq.\ref{zeta0},Eq.\ref{zeta1c},Eq.\ref{zeta1s}に示されるように定義される.

    \begin{figure}[H]
      \centering
      \includegraphics[keepaspectratio, width=0.8\linewidth]{photo/trimanlysis_flow.png}
      \caption{トリム解析の流れ\cite{Yoshizaki_master}}
      \label{fig:trimanlysis_flow}
    \end{figure}

    \begin{equation}
        \zeta = \zeta_{0} + \zeta_{1c}\frac{r}{R}\cos\Psi +  \zeta_{1s}\frac{r}{R}\sin\Psi  \label{zeta}
    \end{equation}
    
    ここで,

    \begin{equation}
        \zeta_{0} = \frac{c_{T}}{2\sqrt{\lambda^2_{0}+ \mu^2}} +  \mu\tan(-\Theta)     \label{zeta0}
    \end{equation}
    
    \begin{equation}
        \zeta_{1c} = \frac{c_{T}}{2\sqrt{\lambda^2_{0}+ \mu^2}}\frac{4}{3}[(1-8\mu^2)\sqrt{1+(\frac{\lambda}{\mu})^2} - \frac{\lambda}{\mu}]  \label{zeta1c}
    \end{equation}

    \begin{equation}
        \zeta_{1s} = \frac{c_{T}}{2\sqrt{\lambda^2_{0}+ \mu^2}}(-2\mu)    \label{zeta1s}
    \end{equation}

    インフローの収束が得られたのちに,胴体に作用する各コンポーネント(メインロータ,テールロータ,胴体,水平/垂直尾翼)からの空力荷重および
    重力が重心位置でバランスした状態での機体座標系における荷重およびモーメント,計6自由度の数式(後述する式 (5-7)参照)における左辺項を
    $\bar{F}=\left \lfloor X, \quad Y, \quad Z, \quad L, \quad M, \quad N \right \rfloor^T$と 表記してトリム変数である
    $\eta=\left\lfloor\Theta, \quad \Psi, \quad \theta_0, \quad \theta_{1 c}, \quad \theta_{1 s}, \quad \theta_{t r}\right\rfloor^T$について
    初期値を$\eta_{0}$としたときに$\bar{F}$を0とするためにTaylorの1次展開式を用いて記載すると

    \begin{equation}
        \bar{\bar{F}}\left(\eta_0+\Delta \eta\right)=\bar{F}\left(\eta_0\right)+\left.\Delta \eta \frac{\partial \bar{F}(\eta)}{\partial \eta}\right|_{\eta=\eta_0}=0
    \end{equation}

    となり,トリム収束解を得るための次のステップとして$\Delta\eta$は\eqref{deltaeta}式の様に表現できる.
 
    \begin{equation}
        \Delta \eta=\left[\left.\frac{\partial \bar{F}(\eta)}{\partial \eta}\right|_{\eta=\eta_0}\right]^{-1} \bar{F}\left(\eta_0\right) \label{deltaeta}
    \end{equation}
 

    ここで,$\left.\frac{\partial \bar{F}(\eta)}{\partial \eta}\right|_{\eta=\eta_0}$は$\bar{F}(\eta)$の
    $\eta=\eta_0$におけるヤコビ行列であり\eqref{jacobian}に示す行列となる

   \begin{equation}
            \left.\frac{\partial \bar{F}(\eta)}{\partial \eta}\right|_{\eta=\eta_0}=\left[\begin{array}{cccccc}
            \frac{\partial X}{\partial \alpha_s} & \frac{\partial X}{\partial \varphi_s} & \frac{\partial X}{\partial \theta_0} & \frac{\partial X}{\partial \theta_{1 c}} & \frac{\partial X}{\partial \theta_{1 s}} & \frac{\partial X}{\partial \theta_{t r}} \\
            \frac{\partial Y}{\partial \alpha_s} & \frac{\partial Y}{\partial \varphi_s} & \frac{\partial Y}{\partial \theta_0} & \frac{\partial Y}{\partial \theta_{1 c}} & \frac{\partial Y}{\partial \theta_{1 s}} & \frac{\partial Y}{\partial \theta_{t r}} \\
            \frac{\partial Z}{\partial \alpha_s} & \frac{\partial Z}{\partial \varphi_s} & \frac{\partial Z}{\partial \theta_0} & \frac{\partial Z}{\partial \theta_{1 c}} & \frac{\partial Z}{\partial \theta_{1 s}} & \frac{\partial Z}{\partial \theta_{t r}} \\
            \frac{\partial L}{\partial \alpha_s} & \frac{\partial L}{\partial \varphi_s} & \frac{\partial L}{\partial \theta_0} & \frac{\partial L}{\partial \theta_{1 c}} & \frac{\partial L}{\partial \theta_{1 s}} & \frac{\partial L}{\partial \theta_{t r}} \\
            \frac{\partial M}{\partial \alpha_s} & \frac{\partial M}{\partial \varphi_s} & \frac{\partial M}{\partial \theta_0} & \frac{\partial M}{\partial \theta_{1 c}} & \frac{\partial M}{\partial \theta_{1 s}} & \frac{\partial M}{\partial \theta_{t r}} \\
            \frac{\partial N}{\partial \alpha_s} & \frac{\partial N}{\partial \varphi_s} & \frac{\partial N}{\partial \theta_0} & \frac{\partial N}{\partial \theta_{1 c}} & \frac{\partial N}{\partial \theta_{1 s}} & \frac{\partial N}{\partial \theta_{t r}}
            \end{array}\right]_{\eta=\eta_0}    \label{jacobian}
    \end{equation}

    \eqref{deltaeta}式で得られる$\Delta\eta$を用いて初期値$\eta_{0}$に$K\Delta\eta(0 < K < 1)$を加算してFig.\ref{fig:trimanlysis_flow}
    に示されるフロー全体が収束するまで,つまりトリム変数$\eta$が収束するまで計算を実施することでトリム計算が完了することになる.

    \section{トリム解析結果}
    \par

    先行研究では解析の対象となる機体は中型機であるSH-60Kとした.
    Table.\ref{table:SH60K}にその諸元を示す.

    \begin{table}[H]
        \centering
        \caption{SH-60K 諸元\cite{Yoshizaki_master}}
        \label{table:SH60K}
        \setlength{\tabcolsep}{5pt}
        \begin{tabular}{|l|l|l|l|}
            \hline \multicolumn{4}{|c|}{SH-60K(三菱重工業株式会社)} \\
            \hline \multicolumn{2}{|c|}{質量} & \multicolumn{2}{|c|}{エンジン} \\
            \hline 空虚質量[kg] & 7,167 & 型式 & T700-IHI-401 \\
            \hline 最大全備質量[kg] & 9,979 & 数量 & 2 \\
            \hline 最大燃料質量[kg] & 2,041 & & \\
            \hline \multicolumn{4}{|c|}{ロータ諸元} \\
            \hline \multicolumn{2}{|c|}{メインロータ} & \multicolumn{2}{|c|}{テールロータ} \\
            \hline 半径[m]& 8.18 &  \multicolumn{2}{|c|}{1.68} \\
            \hline コード長[m]& 0.550 &  \multicolumn{2}{|c|}{0.250} \\
            \hline ソリディティ & 0.086 &  \multicolumn{2}{|c|}{0.19} \\
            \hline ブレード枚数 & 4 &  \multicolumn{2}{|c|}{4} \\
            \hline 回転数[rpm] & 258 &  \multicolumn{2}{|c|}{1,190} \\
            \hline ねじれ角[deg]& 8.8 &  \multicolumn{2}{|c|}{0} \\
            \hline ヒンジオフセット$e/R$ & 0.047 & \multicolumn{2}{|c|}{ヒンジレスロータ} \\
            \hline 
        \end{tabular}
    \end{table}

    \par
    数値解析は機体が30.8 m/s(120 kt)で高速巡航飛行している状態でのトリム解析を実施した.
    特に振動を正確に計算する目的からメインロータブレードは弾性体としてモデル化され,20分割されている.
    各々の要素は断面特性を正確に計算されたモデルである.
    ここで 胴体に関しては剛体としてモデル化している.
    これは詳しくは後述するが,メインロータシャフト,及びメインギアボックスが取りつく構造部位が有意に変形するような胴体の振動モードは
    4/rev(25.0 Hz)からは遠く離れておりメインロータと胴体の連成を考慮に入れなくても問題は生じないためである.
    CAMRAD $\mathrm{II}$によるトリム解析はFig.\ref{fig:sh60k_model}に示すようにベースラインCGから前方に 0.305 m,後方に1.68 mの範囲で,
    また左/右舷 0.912 mの範囲でそれぞれ 0.152 mの刻み幅で設定された4箇所で実施した.
    機体の慣性モーメントはどのCGの位置でもベースラインの値に統一して一定の値を使用した.
    この仮定の下においても胴体の4/rev振動の評価には影響がないから問題はない.
    この詳細についても後述する.

    \begin{figure}[H]
      \centering
      \includegraphics[keepaspectratio, width=0.8\linewidth]{photo/sh60k_model.png}
      \caption{SH-60K 上面図及び側面図\cite{Yoshizaki_master}}
      \label{fig:sh60k_model}
    \end{figure}


    \clearpage 

    \par
    その他の条件については例えばメインロータブレードの弾性変形応答に関わる構造減衰率や自由後流モデルでインフローに影響を与える
    Wakeの周期数などあるが,ここではベースラインで使用している値と同一のものを全ての評価点で使用した.
    また,特に水平尾翼の取付角はベースラインのラジコンヘリコプタと同じく 0 degとした.
    図2-3にトリム変数の結果(ピッチ,ロール,フラッピング角)を示す.
    また,図 2-4にはFig.\ref{fig:sh60k_model}に示される胴体及び水平尾翼の空力作用点に作用する機体座標系における$F_x$及び$F_z$を示す.
    それぞれのグラフの横軸は図 2-1に示される$y$軸を示し,また縦軸は$x$軸を表しており,重心位置が変化した際のそれぞれの値のコンター図となっている.
    それぞれの結果から次に示す所見が得られる.

    \par
    \begin{itemize}
         \item $\Theta$及び$\Phi$は重心位置が後方,右舷に移動すると想定通りそれぞれ増加する一方で,それらの変化を打ち消すように
               $\beta_{c}$及び$\beta_{c}$はそれぞれ増加しメインロータ推力を前方,左舷方向に傾けてトリムを保つ.
         \item 図(2-3)aで$\Theta$が増加すると図(2-4)b~dに示される水平尾翼及び胴体の$-F_z$が増加する.
               このことはメインロータ推力が低下することにつながるため,コーニング角$\beta_{0}$が低下する結果となる(図 2-3) e
         \item 図(2-3)aに示される$\Theta$が5度を超えて増加するに従い,図 (2-4)a~cに示される水平尾翼及び胴体の$-F_x$がそれぞれ増加/減少している.
               $\Theta$は巡航速度とメインロータのダウンウォッシュによる流速により決定される.
               また,重心位置が変化する時,図(2-3)bに示される$\Phi$が-13度から-10度の間で大きく変化するが,
               図(2-4)cの胴体の$-F_x$への影響は比較的小さい.
    \end{itemize}
    

    \section{メインロータハブ位置での振動荷重およびモーメント}
    \par  
    (2.2)にて示したトリム解析結果は重心移動による影響が正しく解析されていることが明確に判断できる結果であり,
    この解析結果をもってメインロータハブ中央位置での振動荷重及びモーメントを評価することにする.
    ここで前述の通り本解析では胴体を剛体として取り扱っているが,Cheng, Q.Y., et al(44)はロータ及び胴体の連成モデルにより
    得られたメインロータブレードの固有振動数は連成を考慮しないモデルにより得られる結果とほぼ同じであることを示している.
    従って胴体を剛体と仮定しても連成モデルと等価であるということができる.
    今回の SH-60Kの解析結果はメインロータブレードの固有振動数は4/revから十分離れていることを示している.(補遺 B)
    更に,Hansford R.E.(45)や Bauchau et al(46)はメインロータ回転面に対して面外(法線)方向の振動荷重は
    メインロータハブの変動応答を考慮しなくても精度よく解析可能であることを示している.
    このことからメインロータブレードのフラッピング運動を解析する際にもメインロータハブの運動を考慮せずに
    振動荷重とモーメントを計算している.

    \par
    図 (2-5)及び図 (2-6)にメインロータハブ中央位置を原点とした図(2-2)に示されている$x_{\mathrm{hub}}$,$y_{\mathrm{hub}}$,$z_{\mathrm{hub}}$
    座標系で表現される4/rev振動荷重およびモーメントの振幅及び位相角のコンタープロットをそれぞれ示している.
    また,Johnson(10)は (1.1)で示したように4枚ブレードのヘリコプタでは4/rev振動成分が卓越することを示している.
    図(2-5)a~fにより それぞれの荷重及びモーメント成分にはそれぞれ異なる重心位置において極小値となる場所が存在することが分かった.
    また,図(2-6)cの$F_{z\mathrm{hub}}$を見るとそのコンタープロットの傾向は図 (2-5)cに示す$F_{z\mathrm{hub}}$のものと類似している.
    振動荷重及びモーメン の振幅が極小となる場所は厳密にはすべて異なっているが,総じて右舷後方の領域の振動が小さいことがこの結果から推定できる.

    \par
    次章では本章で得られた荷重及びモーメントを基に機体の代表的な居住区画であるパイロット席位置での加速度を導出して
    振動レベルの評価を実施することにする.

    

    