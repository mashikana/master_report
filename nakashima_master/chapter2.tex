\chapter{飛行試験}
\label{chap:flight_test}
\begin{flushright}
	\begin{minipage}{0.8\hsize}
		\quad 本章では,ラジコンヘリコプタを用いた飛行試験の概要とその結果を示す.
		      まず,飛行試験に用いたラジコンヘリコプタの機体諸元と飛行条件について示す.
              加えて,重心位置を変更するための錘の位置を説明する.
              そして,飛行試験でのデータ取得方法と計測結果について述べる.
	\end{minipage}
\end{flushright}


    \section{検討対象機体}
    \label{sec:aircraft}
    \par
    Appendix\ref{chap:previous_research}の内容から,$F_{z\mathrm{hub}}$を評価指標とし,重心移動による低振動化の効果の確認を
    飛行試験が可能なラジコンヘリコプタのサイズで行う.
    Fig.\ref{fig:helicopter_model}にラジコンヘリコプタのモデルを示す.
    飛行試験には,4枚ブレードのヒンジレス・ロータを持つラジコンヘリコプタを使用した.
    また,Table.\ref{table:flight_condition}に飛行条件を示す.
    メインロータブレードは弾性ブレードとしてモデル化し,胴体は剛体として扱った.
    搭載するロータはTable.\ref{table:aircraft_specification}に示す諸元のものである.

    \begin{figure}[H]
      \centering
      \includegraphics[keepaspectratio, width=0.7\linewidth]{photo/helicopter_model.png}
      \caption{ラジコンヘリコプタのモデル}
      \label{fig:helicopter_model}
    \end{figure} 

    \begin{table}[H]
        \centering
        \caption{飛行条件}
        \label{table:flight_condition}
        \setlength{\tabcolsep}{5pt}
        \begin{tabular}{|l|l|}
            \hline 飛行形態 &  前進飛行 \\
            \hline 飛行速度 $V_c$ \, [m/s] &  30.0  \\
            \hline メインロータ回転方向 \, [-] & CW \\
            \hline ロータ回転数 \, $\Omega$ \, [RPM]& 1500 \\
            \hline 
        \end{tabular}
    \end{table}

    \begin{table}[H]
        \centering
        \caption{機体諸元}
        \label{table:aircraft_specification}
        \setlength{\tabcolsep}{5pt}
        \begin{tabular}{|l|l|}
            \hline 機体質量 \, $m$ \, [kg] &  11.70 \\
            \hline 半径 \, $R$ \, [m] &  0.786 \\
            \hline ブレードコード長 \, c \, [m] & 0.0617 \\
            \hline 合計ブレード数 \, $b$ [-] & 4 \\
            \hline 平面形  & 矩形 \\
            \hline 翼型 & NACA0012 \\
           \hline  リードラグヒンジオフセット \, $e_0$ \, [m] & CW \\
            \hline 
        \end{tabular}
    \end{table}

    \par
    Fig.\ref{fig:helicopter_cgposition}にシミュレーションと飛行試験で用いたラジコンヘリコプタを示す.
    メインロータの真下に2.7kgの錘を配置した状態を$a$とする.
    このとき,重心はシャフトの真下である.
    また,棒状のアームを胴体から振動が軽減されると想定される左舷後方側に伸ばし,錘をロータの前後方向位置の20\%と40\%の位置に配置し,
    これらをそれぞれ$b$,$c$とする.
    一方,振動が大きくなると予想される右舷後方に対してロータの前後方向位置の40\%の位置に錘を配置した状態を$d$とする.
    Table.\ref{table:cg_position}は,これらのケースにおける重心の位置づけを整理したものである.
  
  
  
    \begin{figure}[htbp]
    \centering
    % 1行目
    \begin{subfigure}[b]{0.4\linewidth}
        \centering
        \includegraphics[width=\linewidth]{photo/helicopter_front.png}
        %\caption{ラジコンヘリコプタ(前面図)}
        \label{fig:helicopter_front}
    \end{subfigure}
    \hfill
    \begin{subfigure}[b]{0.5\linewidth}
        \centering
        \includegraphics[width=\linewidth]{photo/helicopter_up.png}
        %\caption{ラジコンヘリコプタ(上面図)}
        \label{fig:helicopter_up}
    \end{subfigure}

    \vspace{5mm}

    % 2行目
    \begin{subfigure}[b]{0.7\linewidth}
        \centering
        \includegraphics[width=\linewidth]{photo/weight_position.png}
        %\caption{錘の位置}
        \label{fig:weight_position}
    \end{subfigure}
    \hfill
    \begin{subfigure}[b]{0.5\linewidth}
        \centering
        \includegraphics[width=\linewidth]{photo/weight_position_up.png}
        %\caption{錘の位置(4ケース)}
        \label{fig:weight_position_up}
    \end{subfigure}

    \caption{飛行試験用ラジコンヘリコプタと錘取り付け位置}
    \label{fig:helicopter_cgposition}
    \end{figure}


    \begin{table}[H]
        \centering
        \caption{重心の構成}
        \label{table:cg_position}
        \setlength{\tabcolsep}{5pt}
        \begin{tabular}{|l|l|l|l|l|l|}
            \hline         & \multicolumn{2}{|c|}{錘の位置}                   &  \multicolumn{3}{|c|}{メインロータシャフトに対する重心位置[m]} \\
            \hline 重心位置 & ロータ前後方向位置 [\%] & 方位角[deg]              & 前後(後方が正) & 左右(右方が正) & 上下(上方が正)  \\
            \hline a       & \multicolumn{2}{|c|}{メインロータハブ中心の真下}   &  $-1.78\times 10^{-2}$          &  $9.11\times 10^{-4}$           &  $-2.25\times 10^{-1}$ \\
            \hline b       & 20                     &    315                   &  $1.39\times 10^{-2}$           &  $-3.08\times 10^{-2}$          &  $-2.25\times 10^{-1}$ \\
            \hline c       & 40                     &    315                  &  $4.57\times 10^{-2}$           &  $-6.26\times 10^{-2}$          &  $-2.25\times 10^{-1}$ \\
            \hline d       & 40                     &    45                  &  $4.57\times 10^{-2}$           &  $6.44\times 10^{-2}$           &  $-2.25\times 10^{-1}$ \\
            \hline 
        \end{tabular}
    \end{table}


    \clearpage
    
    \section{データ取得方法}
    \label{sec:data_acquisition}
    \ref{sec:aircraft}節で紹介したラジコンヘリコプタには,メインロータシャフトの周辺に加速度/ジャイロセンサーが搭載されており,
    メインロータが1500RPMの一定回転速度で飛行している間,300Hzのサンプリングレートで$x$,$y$,$z$角速度/加速度を
    同時に取得し,SDカードに保存することができる.
    また,ロータ回転速度,GPSの経度,緯度,高度,速度,気圧,温度も計測した.

    \begin{figure}[H]
      \centering
      \includegraphics[keepaspectratio, width=0.6\linewidth]{photo/IMU.png}
      \caption{IMU(Inertial Measurement Unit)}
      \label{fig:IMU}
    \end{figure} 

    \par
    Fig.\ref{fig:cite_altitude_velocity}に示す限られた空域で飛行試験を行うため,約300mの直線飛行経路で往復動作を繰り返し行い,
    最低高度,最高速度でほぼ定常飛行した約3秒間のデータを評価データとして取得した.
    試験は\ref{sec:aircraft}節で述べた4ケースの重心位置で行った.
    飛行試験当日は,曇り空ににわか雨が降るものの,風は穏やかで突風もなく,外乱の影響を受けにくい振動測定に適した気象条件であった.
    ラジコンヘリコプタは,第\ref{chap:previous_research}章で解析・検証したヘリコプタとは異なり,メインロータが時計回りに回転する.
    そのため,振動が最小となる重心位置は,左舷側かつ後方位置であると推定された.
    飛行試験は自律飛行ではなくマニュアル操縦であり,速度の偏差は約30m/sから±3m/sの範囲であった.

    \begin{figure}[htbp]
    \centering
        % 1行目
        \begin{subfigure}[b]{0.55\linewidth}
            \centering
            \includegraphics[width=\linewidth]{photo/nagara_river.png}
            %\caption{長良川}
            \label{fig:nagara_river}
        \end{subfigure}
        \hfill
        \begin{subfigure}[b]{0.4\linewidth}
            \centering
            \includegraphics[width=\linewidth]{photo/altitude_velocity.png}
            %\caption{高度と速度}
            \label{fig:altitude_velocity}
        \end{subfigure}
    \caption{飛行試験場と高度,速度のデータ}
    \label{fig:cite_altitude_velocity}
    \end{figure}

    \clearpage
    \section{計測結果}
    \label{sec:flight_test_result}
    \par
    Table.\ref{table:flight_test_result}に機体構成ごとの試験結果を示す.
    この試験では,重心を通常位置から左舷後方の位置$b$に移動させることで,4/rev振動が26\%減少した.
    重心をさらに左舷後方の位置$c$まで移動させても,通常位置の振動から16\%しか減少しなかった.
    一方,振動が増加すると予想された右舷後方位置$d$では,振動が33\%増加し,予想通りの結果となった.


    \begin{table}[H]
        \centering
        \caption{飛行試験の計測結果}
        \label{table:flight_test_result}
        \setlength{\tabcolsep}{5pt}
        \begin{tabular}{|l|l|l|l|l|l|l|l|l|}
            \hline 重心位置                         & \multicolumn{2}{|c|}{$a$} &  \multicolumn{2}{|c|}{$b$} & $c$ &  \multicolumn{3}{|c|}{$d$} \\
            \hline 計測回数                         & 1回目     & 2回目      & 1回目     &  2回目  &  1回目    & 1回目     & 2回目  &  3回目   \\
            \hline $\Omega$ [RPM]                  & 1494      & 1521       & 1500     & 1490    &  1528     & 1489     & 1494   & 1505  \\
            \hline 気圧 [bar]                      & 1017.08   & 1016.3     & 1015.9   & 1015.1  & 1016.6    & 1015.9   & 1015.5 & 1015.1 \\
            \hline 気温 [°]                        & 18.3      & 18.8       & 15.8     & 16.9    & 16.1      & 14.0     & 14.7   & 15.4   \\
            \hline 高度[m]                         & 13.5      & 18.9       & 17.6     & 27.6    & 10.5      & 10.3     & 12.5   & 18.6   \\
            \hline 速度[m/s]                       & 31.0      & 33.7       & 31.9     & 34.0    & 31.8      & 31.0     & 28.9   & 32.0   \\
            \hline $\Phi$[deg](右舷下げ正)          & 6.1       & 6.1        & -5.7     & -5.0    & 3.5       & 0.6      & 7.4    & 2.5    \\
            \hline $\Theta$[deg](頭上げ正)          & -6.3      & -7.4       & -10.9    & -14.1   & -5.8      & -4.4     & -5.5   & -4.7   \\
            \hline 4/rev $a_z$[$\mathrm{m/s^2}$]   & 1.47      & 1.57       & 1.08     & 1.18    & 1.27      & 1.86     & 1.96   & 2.06   \\
            \hline 平均 4/rev $a_z$[$\mathrm{m/s^2}$] & \multicolumn{2}{|c|}{1.52} & \multicolumn{2}{|c|}{1.13} & 1.27 & \multicolumn{3}{|c|}{1.96}  \\
            \hline 比率                            & \multicolumn{2}{|c|}{1.00} & \multicolumn{2}{|c|}{0.74} & 0.84 & \multicolumn{3}{|c|}{1.33}  \\
            \hline 
        \end{tabular}
    \end{table}
    
    
   