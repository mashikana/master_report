\chapter{考察}

\begin{flushright}
	\begin{minipage}{0.8\hsize}
		\quad 本章では,飛行試験で用いたラジコンヘリコプタについてCAMRAD $\mathrm{II}$による解析結果を示す.
		      まず,飛行試験とによる結果の違いについて述べる.
					次に,ブレード方位角$\Psi$と有効迎角$\alpha$の関係や,有効迎角 $\alpha$と揚力係数$C_l$の関係などのグラフから,
					低振動化に寄与する要素を特定する.
          最後に,$r/R= 0.855$における$\alpha U^2$,$C_l U^2$の振幅の値をFFTにより求め,支配的な周波数成分の組み合わせについて述べる.
        
	\end{minipage}
\end{flushright}

    \section{メインローターブレードの固有振動数とモード形状}
    \label{sec:natural_freqency_mode}
    \par
		表7および図16は,メインローターブレードの固有振動数とモード形状を示している.
		重心移動による固有振動数とモード形状はほぼ変化しないが,ヒンジレスロータを採用すると両者が変化する. 
		フラップとラグはローター半径でスケーリングされ、ピッチはラジアン単位である.
		リード・ラグヒンジを採用した場合,フラッピングモードの振動周波数は4/rev周波数に極めて近接するため,振動の増大に寄与する.

    \section{リードラグヒンジがフラッピングに与える影響}
    \label{sec:flap}
    \par
    前節で述べたように、ラジコンヘリコプターにおいて、羽ばたきモードの振動周波数は4/回転数に極めて近接しているため、振動の増大に寄与している.
		フラップはリード・ラグに関連するが,ラジコンヘリコプターにはフラッピングヒンジが存在しないため,リード・ラグヒンジにおけるブレード拘束に自由度が生まれる.
		したがって,リード・ラグヒンジにおける拘束条件をパラメータとして振動解析を実施した.
		図17に各caseにおけるフラップを示す.
	  重心移動により,フラップの1/rev成分が支配的となった.
		リード・ラグヒンジを使用することで,フラップの最大値が減少する.
		図18は、各ケースにおける4/revの$F_{z\mathrm{hub}}$を示している.
		フラップの最大値の抑制も4/revの$F_{z\mathrm{hub}}$の低減に重要であるが,ブレードの固有振動数をずらすことによる効果の方がより顕著である.


    


