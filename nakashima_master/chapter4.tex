\chapter{考察}

\begin{flushright}
	\begin{minipage}{0.8\hsize}
		\quad 本章では,最適なブレード設計を行うために考慮すべき点について述べ,その効果について考察する.
		    まず,CFD解析より,空気力によるねじり下げモーメントについて述べる.遠心力によるねじり下げモーメントの重要性について考察する.
        次に翼型がねじり下げモーメントに与える影響について考察する.最後に,設計時のねじり下げがねじり下げモーメントに与える影響について考察する.
        
	\end{minipage}
\end{flushright}



\section{空力によるねじり下げモーメント}
\par
Figにブレードピッチ角を変化させた場合のねじり下げモーメントの変化を示す.
このグラフではねじり下げ方向を正としている.どの角速度においてもWB2が最も大きくなっている.またどの角速度においても
ピッチ角の変化に対するねじり下げモーメントの変化は同様の傾向がみられる.WB2については,ピッチ角が上がるにつれて
ねじり下げモーメントが減少している.NB2については,15度まではほぼ変わらないが,20度になると約1.5倍ほど増加している.
NB4はピッチ角の変化に対してはねじり下げモーメントがほぼ変化しないことがわかる.また,角速度が約1.5倍になると,ねじり下げモーメントが
約2.25倍になっており,ねじり下げモーメントは角速度の2乗に比例していることがわかる.したがって,空気力によるねじり下げモーメントでは
ピッチ角よりもロータの角速度による影響が大きいといえる.

\section{翼型の影響}

\par
本研究では,これまで翼型をOAF117として検討を行ってきた.これは,ねじり下げを有するブレードであるOAF117が対称翼のNACA0009よりも空力性能が良いことが先行研究
より得られており,最適なブレード設計においては空力性能が良いことは重要といえるからである.しかしながら,翼型が遠心力によるねじり下げモーメントに与える影響を考慮することで,
より最適なブレードの設計が可能であると考える.本節では,そのような経緯から,翼型の影響について考察する.
Fig.に翼型が遠心力によるねじり下げモーメントに与える影響について示す.
グラフの赤い部分がねじり下げに影響がある部分で,青い部分がねじり上げに影響がある部分である.対称翼に関しては,ピッチが上がるにつれてねじり下げモーメントが増加していくと考えられる.
一方でOAF117のような非対称翼については,ねじり上げに影響がある部分が対称翼に比べて多く,高いピッチにおいても遠心力によるねじり下げモーメントを低減できると考えられる.
また,設計時にねじり下げを付けることで回転軸から距離が離れている部分も,ねじり下げモーメントの増加を抑えることが出来ると考えられる.したがって,非対称翼で,ねじり下げを有しているブレード形状が
最適であると考えられる.

