\chapter{考察}

\begin{flushright}
	\begin{minipage}{0.8\hsize}
		\quad 本章では,飛行試験で用いたラジコンヘリコプタについてCAMRAD $\mathrm{II}$による解析結果を示す.
		      まず,飛行試験とによる結果の違いについて述べる.
					次に,ブレード方位角$\Psi$と有効迎角$\alpha$の関係や,有効迎角 $\alpha$と揚力係数$C_l$の関係などのグラフから,
					低振動化に寄与する要素を特定する.
          最後に,$r/R= 0.855$における$\alpha U^2$,$C_l U^2$の振幅の値をFFTにより求め,支配的な周波数成分の組み合わせについて述べる.
        
	\end{minipage}
\end{flushright}

    \section{メインロータブレードの固有振動数とモード形状}
    \label{sec:natural_frequency_mode}
    \par
		Table.\ref{table:natural_frequency}およびFig.\ref{fig:mode_shape_lag}にメインロータブレードの固有振動数とモード形状を示す.
		なお,フラップとラグの変位はローター半径で無次元化され、ピッチ角はラジアンである.
		リード・ラグヒンジを採用した場合,フラッピングモードの振動周波数は4/rev周波数に極めて近接するため,振動の増大に寄与する.
    フラップはリード・ラグに関連するが,ラジコンヘリコプタにはフラッピングヒンジが存在しないため,リード・ラグヒンジにおけるブレード拘束に自由度が生まれる.
		したがって,リード・ラグヒンジにおける拘束条件をパラメータとして振動解析を実施した.
		Table.\ref{table:natural_frequency_less}およびFig.\ref{fig:mode_shape_less}に,ヒンジレスロータを搭載したラジコンヘリコプタにおける
		メインロータブレードの固有振動数とモード形状を示す.
		ヒンジレスロータを採用すると,固有振動数とモード形状が変化することが確認できる. 

    \begin{table}[H]
        \centering
        \caption{MRBの固有振動数解析結果}
        \label{table:natural_frequency}
        \setlength{\tabcolsep}{5pt}
        \begin{tabular}{|l|l|l|l|}
            \hline No. &   モード形状   &   固有振動数    &  モード形     \\
            \hline 1   &   ラグ1次      &   0.455/rev    &  Fig.\ref{fig:mode_first_a_lag}   \\
            \hline 2   &   フラップ1次  &   1.257/rev    &  Fig.\ref{fig:mode_second_a_lag}    \\
            \hline 3   &   フラップ2次  &   4.058/rev    &  Fig.\ref{fig:mode_third_a_lag}   \\
            \hline 4   &   ピッチ1次    &   4.283/rev    &  Fig.\ref{fig:mode_fourth_a_lag}    \\
            \hline 
        \end{tabular}
    \end{table}

		\begin{figure}[htbp]
				\centering

				\begin{subfigure}{\linewidth}
						\centering
						\includegraphics[height=0.2\textheight]{photo/mode_first_a_lag.png}
						\caption{ラグ1次}
						\label{fig:mode_first_a_lag}
				\end{subfigure}

				\begin{subfigure}{\linewidth}
						\centering
						\includegraphics[height=0.2\textheight]{photo/mode_second_a_lag.png}
						\caption{フラップ1次}
						\label{fig:mode_second_a_lag}
				\end{subfigure}

				\begin{subfigure}{\linewidth}
						\centering
						\includegraphics[height=0.2\textheight]{photo/mode_third_a_lag.png}
						\caption{フラップ2次}
						\label{fig:mode_third_a_lag}
				\end{subfigure}

				\begin{subfigure}{\linewidth}
						\centering
						\includegraphics[height=0.2\textheight]{photo/mode_fourth_a_lag.png}
						\caption{ピッチ1次}
						\label{fig:mode_fourth_a_lag}
				\end{subfigure}

				\caption{ラジコンヘリコプタのMRBのモード形}
				\label{fig:mode_shape_lag}
		\end{figure}



    \begin{table}[H]
        \centering
        \caption{MRBの固有振動数解析結果(ヒンジレスロータ)}
        \label{table:natural_frequency_less}
        \setlength{\tabcolsep}{5pt}
        \begin{tabular}{|l|l|l|l|}
            \hline No. &   モード形状   &   固有振動数    &  モード形     \\
            \hline 1   &   フラップ1次  &   1.180/rev    &  Fig.\ref{fig:mode_first_a_less}   \\
            \hline 2   &   ラグ1次      &   1.391/rev    &  Fig.\ref{fig:mode_second_a_less}    \\
            \hline 3   &   ピッチ1次    &   3.741/rev    &  Fig.\ref{fig:mode_third_a_less}    \\
            \hline 4   &   ピッチ2次    &   4.096/rev    &  Fig.\ref{fig:mode_fourth_a_less}    \\
            \hline 
        \end{tabular}
    \end{table}

		\begin{figure}[htbp]
				\centering

				\begin{subfigure}{\linewidth}
						\centering
						\includegraphics[height=0.2\textheight]{photo/mode_first_a_less.png}
						\caption{フラップ1次}
						\label{fig:mode_first_a_less}
				\end{subfigure}

				\begin{subfigure}{\linewidth}
						\centering
						\includegraphics[height=0.2\textheight]{photo/mode_second_a_less.png}
						\caption{ラグ1次}
						\label{fig:mode_second_a_less}
				\end{subfigure}

				\begin{subfigure}{\linewidth}
						\centering
						\includegraphics[height=0.2\textheight]{photo/mode_third_a_less.png}
						\caption{ピッチ1次}
						\label{fig:mode_third_a_less}
				\end{subfigure}

				\begin{subfigure}{\linewidth}
						\centering
						\includegraphics[height=0.2\textheight]{photo/mode_fourth_a_less.png}
						\caption{ピッチ2次}
						\label{fig:mode_fourth_a_less}
				\end{subfigure}

				\caption{ラジコンヘリコプタのMRBのモード形(ヒンジレスロータ)}
				\label{fig:mode_shape_less}
		\end{figure}


		\clearpage
    \section{リードラグヒンジがフラッピング運動に与える影響}
    \label{sec:flap}
    \par
		Fig.\ref{fig:comparison_sz}に示すように,リードラグヒンジの有無によって,空力である$S_{z}$には大きな変化は見られなかった.
		一方,Fig.\ref{fig:comparison_fz}に示すように,ブレード4枚分の$F_{z}$には約10Nほどの違いが見られる.
		Fig.\ref{fig:flap}に$r/R = 0.855$におけるフラップの変位を示す.
	  重心移動により,フラップの1/rev成分が支配的となった.
		リード・ラグヒンジを使用することで,フラップの最大値が減少する.
		これは,フラップの変位が増え,その結果として慣性力が増大したためと考えられる.
		図18は,各ケースにおける4/revの$F_{z\mathrm{hub}}$を示している.
		
    \begin{figure}[htbp]
    \centering
        \begin{subfigure}[b]{0.45\linewidth}
            \centering
            \includegraphics[width=\linewidth]{photo/Sz_a.png}
            \caption{with lead-lag hinge}
            \label{fig:Sz_a}
        \end{subfigure}
        \hfill
        \begin{subfigure}[b]{0.45\linewidth}
            \centering
            \includegraphics[width=\linewidth]{photo/Sz_a_less.png}
            \caption{without lead-lag hinge}
            \label{fig:Sz_a_less}
        \end{subfigure}
    \caption{$z$方向空力荷重$S_z$の比較}
    \label{fig:comparison_sz}
    \end{figure}

    \begin{figure}[htbp]
    \centering
        \begin{subfigure}[b]{0.45\linewidth}
            \centering
            \includegraphics[width=\linewidth]{photo/Fz_a.png}
            \caption{with lead-lag hinge}
            \label{fig:Fz_a}
        \end{subfigure}
        \hfill
        \begin{subfigure}[b]{0.45\linewidth}
            \centering
            \includegraphics[width=\linewidth]{photo/Fz_a_less.png}
            \caption{without lead-lag hinge}
            \label{fig:Fz_a_less}
        \end{subfigure}
    \caption{$z$方向荷重$F_z$の比較}
    \label{fig:comparison_fz}
    \end{figure}

    \begin{figure}[htbp]
    \centering
        \begin{subfigure}[b]{0.45\linewidth}
            \centering
            \includegraphics[width=\linewidth]{photo/flap_0855_a.png}
            \caption{with lead-lag hinge}
            \label{fig:flap_0855_a}
        \end{subfigure}
        \hfill
        \begin{subfigure}[b]{0.45\linewidth}
            \centering
            \includegraphics[width=\linewidth]{photo/flap_0855_a_less.png}
            \caption{without lead-lag hinge}
            \label{fig:flap_0855_a_less}
        \end{subfigure}
    \caption{$r/R = 0.855$でのフラップ}
    \label{fig:flap}
    \end{figure}
    

    \begin{figure}[htbp]
    \centering
        \begin{subfigure}[b]{0.45\linewidth}
            \centering
            \includegraphics[width=\linewidth]{photo/inertia_4blade_a.png}
            \caption{with lead-lag hinge}
            \label{fig:inertia_4blade_a}
        \end{subfigure}
        \hfill
        \begin{subfigure}[b]{0.45\linewidth}
            \centering
            \includegraphics[width=\linewidth]{photo/inertia_4blade_a_less.png}
            \caption{without lead-lag hinge}
            \label{fig:inertia_4blade_a_less}
        \end{subfigure}
    \caption{方位角$\Psi$と慣性力の関係(4ブレード)}
    \label{fig:inertia_4blade}
    \end{figure}
    

