\chapter{結論}
\par
第1章では,シングルロータヘリコプタの現状と運用上の課題であるN/revの振動荷重について説明した.
次に,低振動化への従来の取り組みを示した.
そして,本論文の提案手法である重心移動によるシングルロータヘリコプタの低振動化について述べた.
最後に本研究で使用する解析ツールについて説明した.

第2章では,ラジコンヘリコプタを用いた飛行試験の概要とその結果を示した.
まず,飛行試験に用いたラジコンヘリコプタの機体諸元と飛行条件について示した.
加えて,重心位置を変更するための錘の位置を説明した.
次に,飛行試験でのデータ取得方法と計測結果について述べた.

第3章では,飛行試験で用いたラジコンヘリコプタについてCAMRAD $\mathrm{II}$による解析結果を示した.
まず,飛行試験とによる結果の違いについて述べた.
次に,ブレード方位角$\Psi$と有効迎角$\alpha$の関係や,有効迎角 $\alpha$と揚力係数$C_l$の関係などのグラフから,低振動化に寄与する要素を特定した.
最後に,$r/R= 0.855$における$\alpha U^2$,$C_l U^2$の振幅の値をFFTにより求め,支配的な周波数成分の組み合わせについて述べた.

第4章では,リード・ラグヒンジが振動特性に与える影響について考察した.
まず,リード・ラグヒンジの有無によるメインロータブレードの固有振動数とモード形状の違いについて述べる.
次に,数式的に$F_z$と$S_z$の違いについて述べた.
最後に,フラップの変位や慣性力のグラフから,ハブのコンフィグレーションによって$F_{z\mathrm{hub}}$に差が出る原因を考察した.

本論文では,ラジコンヘリコプタを用いて,通常の重心位置を含む4つのケースで重心位置を移動させ振動測定を行った.
先行研究の結果に基づき,振動が減少する位置は主ロータの回転方向に依存すると仮説を立てた.
具体的には,反時計回り回転の場合は右舷後方,時計回り回転の場合は左舷後方である.
時計回りに回転するブレードを備えたラジコンヘリコプタを用いて飛行試験を実施した.
重心位置を後方左側に移動させると4/rev振動が減少した一方,後方右側に移動させると4/rev振動が増加した.
いずれの場合も観測された変化率は約30\%であった.
CAMRAD $\mathrm{II}$を用いた解析により以下のことが明らかになった:
\begin{itemize}
    \item 4/rev振動は,有効迎角の4/rev成分,揚力係数,および流入速度の二乗の定常成分において支配的である.
    \item 有効迎角と揚力係数の方位角に対する変動は,揚力傾斜の差とほぼ等しく,翼の非定常効果は小さいことを示している.
    \item ローターブレードの有効迎角は失速角より小さく,ダイナミックストールが4/rev振動に及ぼす影響は認められなかった.
    \item ブレードの最大有効迎角を低減し,ブレードと渦の干渉効果を最小化することで,航空機の4/rev振動を抑制できる.
\end{itemize}
この手法はシングルロータヘリコプタの振動を低減し,振動に敏感なセンサーを必要とする計測への適用を可能とする.
ただし,重心移動は離着陸時の姿勢角を増加させる.
この問題と4/revの振動抑制の両方を考慮し,適切な重心移動量を決定する必要がある.

また,ハブのコンフィグレーションを変えることによって,フラップの4/rev成分ないしは慣性力に影響を及ぼし,
$F_{z\mathrm{hub}}$の4/rev成分を低減することが可能である.

