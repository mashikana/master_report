\chapter{結論}
\par
第1章ではマルチロータ機の現状と運用上の課題である外乱に対する安定性の低さと制御の応答の重要性について実例をもとに説明した.
そして,外乱に対する制御の応答の遅さを解決するための対策手法として本研究の目的とそれによって達成される利点を述べるた.
最後に本研究の具体的な流れについて説明する.


第2章では可変ピッチロータ搭載型のマルチロータ機に関して,遠心力によるねじり下げ発生の要因と解析対象の機体について示した.また,検討するブレード形状に
ついてその特徴を述べた.


第3章ではねじり下げモーメントを相殺するカウンターウェイトの検討と,消費パワーの計算結果を示した.
まず,ホバリング状態における必要推力を決定し,ホバリング時の性能指数を用いて,ピッチ角,角速度を決定した.
トリム時に発生する遠心力によるねじり下げモーメントを相殺するカウンターウェイトを搭載し,推力を増加させた際の
ねじり下げモーメントの変化について述べた.最後に,ねじり下げモーメントによって発生する消費パワーを計算し,
サーボモータに与える負荷を検討した.

第4章では最適なブレード設計を行うために考慮すべき点について述べ,その効果について考察した.
CFD解析により,空気力によるねじり下げモーメントを考慮し,遠心力によるねじり下げモーメントとの比較を行った.
また,翼型や設計時のブレードのねじり下げがモーメントに与える影響について検討した.


今後の課題としては,カウンターウェイトの搭載による,機体形状の制約についての検討,ロータ個数の変更による,ねじり下げモーメントの低減の検討などが考えられる.
