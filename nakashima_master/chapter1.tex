\chapter{序論}
\label{chap:intro}
\begin{flushright}
	\begin{minipage}{0.8\hsize}
		\quad 本章では,本論文の研究背景と研究目的について説明する.
		      まず,シングルロータヘリコプタの現状と運用上の課題であるN/revの振動荷重について説明する.
              次に,低振動化への従来の取り組みを示す.
              そして,本論文の提案手法である重心移動によるシングルロータヘリコプタの低振動化について述べる.
              最後に本研究で使用する解析ツールについて説明する.
	\end{minipage}
\end{flushright}
    \section{研究背景}
    \label{sec:research_background}
    \par
    航空機は一般に揚力を利用して飛行するが,固定翼機とは異なり,回転翼機はホバリングや鉛直方向の移動が可能である.
    この特性から,災害救助活動やドクターヘリによる患者輸送など,多様な応用が可能となっている.
    固定翼機における揚力は主翼によって発生するのに対し,回転翼機では一定角速度で回転する複数の翼(ブレード)によって揚力が発生する.
    これらのブレードは,メインロータとして機体の運動を制御する主機構を構成しており,通常2枚以上で構成される.
    メインロータブレードのピッチ角は,スワッシュプレートと呼ばれる機構を介して定常的にも周期的にも変更可能である.
    この機構により,機体の上昇・降下・前進・横進といった各方向の運動制御が行われる.
    しかしながら,メインロータが回転する際には必然的に機体に対してトルクが作用する.
    このトルクを打ち消すために,回転翼機には様々な様式がある.
    代表的な構成として,単一のメインロータと機体尾部のテールロータによってトルクを制御するシングルロータヘリコプタが挙げられる.
    一例として,SH-60K哨戒ヘリコプタ(Fig.\ref{SH-60K(シングルロータ)})が挙げられる.
    さらに,前後方向に配置された2つのロータが互いのトルクを打ち消すことで機体を安定化させるタンデムロータヘリコプタが知られている.
    この形式には,CH-47J(Fig.\ref{CH-47J(タンデムロータ)})のような機体がある.
    タンデムロータ機は,ロータが機体前後方向に存在するため,胴体の重心移動に対してロバスト性を有する特徴を持つ.
    また,二重反転式ロータヘリコプタとして,X2(Fig.\ref{X2(二重反転式ロータ)})等の上下二段で反対方向に回転するロータを備えた形式が存在し,
    テールロータを不要としている.
    さらに,二重反転ロータの機構を簡素化し整備性を向上させた交差反転式ロータヘリコプタとして,
    K-MAX(Fig.\ref{K-MAX(交差反転式ロータ)})のような機体も存在する.(Fig.\ref{fig:helicopter_types})
    これらいずれの回転翼機においても,前進飛行時にはロータ面が飛行速度ベクトルに対してほぼ平行となるエッジワイズフライトとなる.
    この状態は固定翼機におけるプロペラ運動とは大きく異なり,回転翼まわりの空気の流れに非軸対称性を生じさせ,振動の発生要因となる.
    例えば,シングルロータの上面図をFig.\ref{fig:forwardflight}に示す.
    メインロータブレードに作用する空力荷重は,ブレードの位置を表すアジマス角$\Psi$に依存して異なる.
    特に,前進側ブレード($\Psi$ = 90 deg)と後退側ブレード($\Psi$ = 270 deg)では,ブレードに作用する相対速度が大きく異なる.
    前進側/後退側ブレードのRotor Station $r$ 位置の微小翼素が受けるロータ回転面内流体速度は次の式で表される.
    (ロータ回転面は$V_{c}$に対して平行とする.)

    \begin{equation}
        V = V_{c} + r\Omega (\Psi = 90 \mathrm{deg})
    \end{equation}
    
    \begin{equation}
        V = -V_{c} + r\Omega (\Psi = 270 \mathrm{deg})
    \end{equation}

    \begin{figure}[htbp]
    \centering

    % 1行目
    \begin{subfigure}[b]{0.45\linewidth}
        \centering
        \includegraphics[width=\linewidth]{photo/sh60k.jpg}
        \caption{SH-60K(シングルロータ)}
        \label{SH-60K(シングルロータ)}
    \end{subfigure}
    \hfill
    \begin{subfigure}[b]{0.45\linewidth}
        \centering
        \includegraphics[width=\linewidth]{photo/CH_47J.jpg}
        \caption{CH-47J(タンデムロータ)}
        \label{CH-47J(タンデムロータ)}
    \end{subfigure}

    \vspace{5mm}

    % 2行目
    \begin{subfigure}[b]{0.45\linewidth}
        \centering
        \includegraphics[width=\linewidth]{photo/X2.jpg}
        \caption{X2(二重反転式ロータ)}
        \label{X2(二重反転式ロータ)}
    \end{subfigure}
    \hfill
    \begin{subfigure}[b]{0.45\linewidth}
        \centering
        \includegraphics[width=\linewidth]{photo/KMAX.jpg}
        \caption{K-MAX(交差反転式ロータ)}
        \label{K-MAX(交差反転式ロータ)}
    \end{subfigure}

    \caption{ヘリコプタの様式\supcite{SH60K,CH47J,X2,KMAX}}
    \label{fig:helicopter_types}
    \end{figure}


    \begin{figure}[H]
      \centering
      \includegraphics[keepaspectratio, width=0.8\linewidth]{photo/forwardflight.png}
      \caption{前進飛行時のシングルロータ (上面図 )\supcite{Prouty2001}}
      \label{fig:forwardflight}
    \end{figure}


    \par
    高速前進飛行時のヘリコプタにおいては,$N$/rev振動($N$:メインロータのブレード枚数)が卓越することが知られている.
    この$N$/rev振動は,メインロータブレードに作用する回転面外方向成分の荷重およびモーメントのうち,$N$/rev成分のみが機体に直接伝達されること,
    ならびに回転面内方向成分に含まれる$(N-1)$/revおよび$(N+1)$/revの荷重とモーメントが,$N$/rev成分として機体に伝達することに起因するものである.
    メインロータブレードのルート部においては,1/rev以上の高調波成分の荷重およびモーメントが発生する.
    これは,回転翼機が前進飛行する際,メインロータ回転面へのインフロー分布が非一様となり,回転する各ブレード翼素の迎角$\alpha$に影響を与えるためである.
    その結果として生じるルート部の荷重およびモーメントは,メインロータブレードが弾性体であるがゆえに,N/rev成分($N=1,2,\dots$)近傍に存在する固有振動モードを励起する.
    この励起により,1/rev以上の高調波成分を含む荷重およびモーメントが発生することとなる.
    通常,初期の開発設計段階においては,これらの影響を低減するよう配慮した設計が行われているものの,その発生自体を完全に抑制することは困難である.
    特に回転面外方向の振動荷重については,回転する各ブレードに個別に発生する$N$/rev荷重が,胴体側において$N$/rev荷重として伝達される.
    このため,機体に伝達される荷重には,主として1/rev成分として観測される荷重の中に高調波成分が含まれる形で現れる.
    一方,$N$枚のメインロータブレードが,厳密に管理された製造工程により,形状,重量および剛性のいずれにおいても極めて均一な品質を有していると仮定するならば,
    $N$/rev成分以外の高調波成分は,機体座標系へ伝達される過程で互いに打ち消し合い,結果として極めて小さな値となり,実用上問題とならない.
    ここで N/rev 振動とは,メインロータハブ中央位置における$x_{\mathrm{hub}}$,$y_{\mathrm{hub}}$,$z_{\mathrm{hub}}$座標系で定義される$N$/rev成分の
    $F_{x\mathrm{hub}}$,$F_{y\mathrm{hub}}$,$F_{z\mathrm{hub}}$,$M_{x\mathrm{hub}}$,$M_{y\mathrm{hub}}$,$M_{z\mathrm{hub}}$に起因する振動を指す.
    これらの$N$/rev成分は,その荷重またはモーメントがインプレーンであるか,アウトオブプレーンであるかによって,メインロータ回転系における振動成分が異なる.
    (Table.\ref{table:Sz_to_Fzhub}~Table.\ref{table:NL_to_Mzhub})
    回転翼機における振動源としては,このほかにもシングルロータ機におけるテールロータ由来の振動,エンジンおよび駆動系統からの振動,
    さらには操縦系統からのトランジェントな入力やガスト応答などが挙げられる.
    しかしながら,これらはいずれもメインロータ由来の$N$/rev振動と比較して十分に小さいため,本論文では議論の対象としない.

    \par
    メインロータハブ中央位置における荷重およびモーメントをTable.\ref{table:Sz_to_Fzhub}~Table.\ref{table:NL_to_Mzhub}に示す.
    これらの表は,回転するブレードが空力荷重を受けて結果的に固定側である胴体側に伝わる荷重をまとめており,回転系でのブレードルート部における荷重を
    Fig.\ref{fig:MRHFM}に示す.
    
    \begin{figure}[H]
      \centering
      \includegraphics[keepaspectratio, width=0.5\linewidth]{photo/MRHFM.png}
      \caption{回転系におけるメインロータブレードルート部荷重 およびモーメント}
      \label{fig:MRHFM}
    \end{figure}
    
    \begin{equation}
        \begin{aligned}
            & S_z=S_{z0}+S_{znc} \cos n \Psi_m+S_{zns} \sin n \Psi_m(n=1, \quad 2 \cdots, \quad m=1, \quad \cdots N) \\
            & S_x=S_{x0}+S_{xnc} \cos n \Psi_m+S_{xns} \sin n \Psi_m(n=1, \quad 2 \cdots, \quad m=1, \quad \cdots N) \\
            & S_r=S_{r0}+S_{rnc} \cos n \Psi_m+S_{rns} \sin n \Psi_m(n=1,2 \cdots, \quad m=1, \quad \cdots N) \\
            & N_F=N_{F0}+N_{Fnc} \cos n \Psi_m+N_{Fns} \sin n \Psi_m(n=1,2 \cdots, \quad m=1, \quad \cdots N) \\
            & N_L=N_{L0}+N_{Lnc} \cos n \Psi_m+N_{Lns} \sin n \Psi_m(n=1, \quad 2 \cdots, \quad m=1, \quad \cdots N)  \label{SandN}
        \end{aligned}
    \end{equation}

    ここでEq.\eqref{SandN}に示すように,$N$本ブレードを有するヘリコプタにおいて,アジマス角 $\Psi_m$ に位置する$m$番目ブレードのルート部に作用する荷重は,
    一般に定常成分,1/rev成分,およびそれらの高調波成分から構成される.
    例えば$S_z$については,各周波数成分を列挙することで,Table.\ref{table:Sz_to_Fzhub}の左列に示される成分の総和として表される.
    なお,6/rev以上の高調波成分は省略する.
    また,Table.\ref{table:Sz_to_Fzhub}は$F_{z\mathrm{hub}}$に関する結果を示しており,メインロータ回転面内の荷重である
    $F_{x\mathrm{hub}}$, $F_{y\mathrm{hub}}$ については Table.\ref{table:SxSr_to_Fxhub} および Table.\ref{table:SxSr_to_Fyhub} に示す.
    さらに,$M_{x\mathrm{hub}}$,$M_{y\mathrm{hub}}$,および$M_{z\mathrm{hub}}$についてはTable.\ref{table:NF_to_Mxhub}~Table.\ref{table:NL_to_Mzhub} に示す.
    Table.\ref{table:Sz_to_Fzhub}~Table.\ref{table:NL_to_Mzhub}では,左列に周波数成分を整理して示しており,
    回転しているブレードルート荷重が,回転していないMRH座標系へどのように伝達されるかをまとめている.
    これらはすべてブレード枚数$N=4$の場合の結果である.
    胴体系に伝達された後の荷重およびモーメントについては,Table.\ref{table:Sz_to_Fzhub}およびTable.\ref{table:NL_to_Mzhub}に示されるように,
    アウトオブプレーンの荷重$F_{z\mathrm{hub}}$およびモーメント$M_{z\mathrm{hub}}$ は,定常成分および4/rev成分が周波数変調することなく,回転系から胴体系へ伝達される.
    また, 4/revの整数倍以外の高調波成分は0となる.
    一方,インプレーンの荷重およびモーメントについては,Table.\ref{table:SxSr_to_Fyhub}~Table.\ref{table:NF_to_Myhub}に示されるように,
    回転系における3/revおよび5/rev成分が変調し,胴体系において4/rev成分として伝達する.
    また,回転系の1/rev成分は定常成分として伝達され,それ以外の高調波成分は0となる.
    なお,伝達後のアジマス角$\Psi$は$m=1$のアジマス角を基準としている.
    各荷重およびモーメント成分に対する周波数変調の詳細についてはRef.\cite{johnson_2013}に示されている.
    特に,変調後にキャンセルされず0とならない場合についてはAppendix\ref{chap:StoF}に示す.

    \begin{table}[H]
        \centering
        \caption{$S_z$各周波数成分の胴体系伝達後の$F_{z\mathrm{hub}}$}
        \label{table:Sz_to_Fzhub}
        \setlength{\tabcolsep}{5pt}
        \begin{tabular}{|c|c|c|} \hline
            周波数 &  $S_z$荷重成分 &   MRH荷重$F_{z\mathrm{hub}}$ N=4 \\ \hline
            定常 &  $S_{z0}$ & $4S_{z0}$ \\ \hline
            \raisebox{-1.2ex}{1/rev} & $S_{z1c}\cos\Psi_m$ & $0$ \\   \cline{2-3}
                                   & $S_{z1s}\sin\Psi_m$ & $0$ \\ \hline
            \raisebox{-1.2ex}{2/rev} & $S_{z2c}\cos 2\Psi_m$  & $0$ \\   \cline{2-3}
                                   & $S_{z2s}\sin 2\Psi_m$  & $0$ \\ \hline
            \raisebox{-1.2ex}{3/rev} & $S_{z3c}\cos 3\Psi_m$  & $0$ \\   \cline{2-3}
                                   & $S_{z3s}\sin 3\Psi_m$  & $0$ \\ \hline
            \raisebox{-1.2ex}{4/rev} & $S_{z4c}\cos 4\Psi_m$  & $4S_{z4c}\cos 4\psi$ \\   \cline{2-3}
                                   & $S_{z4s}\sin 4\Psi_m$  & $4S_{z4s}\sin 4\Psi$ \\ \hline
            \raisebox{-1.2ex}{5/rev} & $S_{z5c}\cos 5\Psi_m$  & $0$ \\    \cline{2-3}
                                   & $S_{z5s}\sin 5\Psi_m$  & $0$ \\ \hline
            \raisebox{-1.2ex}{6/rev} & $S_{z6c}\cos 6\Psi_m$  & $0$ \\    \cline{2-3}
                                   & $S_{z6s}\sin 6\Psi_m$  & $0$ \\ \hline
        \end{tabular}
    \end{table}


    \begin{table}[H]
        \centering
        \caption{$S_x$,$S_r$各周波数成分の胴体系伝達後の$F_{x\mathrm{hub}}$}
        \label{table:SxSr_to_Fxhub}
        \setlength{\tabcolsep}{5pt}
        \begin{tabular}{|c|c|c|c|} \hline
            周波数 &  $S_x$荷重成分 & $S_r$荷重成分 &  MRH荷重$F_{x\mathrm{hub}}$ N=4 \\ \hline
            定常 &  $S_{x0}$ &  $S_{r0}$ & $0$ \\ \hline
            \raisebox{-1.2ex}{1/rev} & $S_{x1c}\cos\Psi_m$ & $S_{r1c}\cos\Psi_m$ & $2S_{r1c}$ \\   \cline{2-4}
                                   & $S_{x1s}\sin\Psi_m$ & $S_{r1s}\sin\Psi_m$ & $2S_{r1s}$ \\ \hline
            \raisebox{-1.2ex}{2/rev} & $S_{x2c}\cos 2\Psi_m$  & $S_{r2c}\cos 2\Psi_m$  & $0$ \\   \cline{2-4}
                                   & $S_{x2s}\sin 2\Psi_m$  & $S_{r2s}\sin 2\Psi_m$  & $0$ \\ \hline
            \raisebox{-1.2ex}{3/rev} & $S_{x3c}\cos 3\Psi_m$  & $S_{r3c}\cos 3\Psi_m$  & $2S_{x3c}\sin 4\Psi + 2S_{r3c}\cos 4\Psi$ \\   \cline{2-4}
                                   & $S_{x3s}\sin 3\Psi_m$  & $S_{r3s}\sin 3\Psi_m$  & $-2S_{x3s}\cos 4\Psi + 2S_{r3s}\sin 4\Psi$ \\ \hline
            \raisebox{-1.2ex}{4/rev} & $S_{x4c}\cos 4\Psi_m$  & $S_{r4c}\cos 4\Psi_m$  & $0$ \\   \cline{2-4}
                                   & $S_{x4s}\sin 4\Psi_m$  & $S_{r4s}\sin 4\Psi_m$  & $0$ \\ \hline
            \raisebox{-1.2ex}{5/rev} & $S_{x5c}\cos 5\Psi_m$  & $S_{r5c}\cos 5\Psi_m$  & $-2S_{x5c}\sin 4\Psi + 2S_{r5c}\cos 4\Psi$ \\    \cline{2-4}
                                   & $S_{x5s}\sin 5\Psi_m$  & $S_{r5s}\sin 5\Psi_m$  & $2S_{x5s}\cos 4\Psi + 2S_{r5s}\sin 4\Psi$ \\ \hline
            \raisebox{-1.2ex}{6/rev} & $S_{x6c}\cos 6\Psi_m$  & $S_{r6c}\cos 6\Psi_m$  & $0$ \\    \cline{2-4}
                                   & $S_{x6s}\sin 6\Psi_m$  & $S_{r6s}\sin 6\Psi_m$  & $0$ \\ \hline
        \end{tabular}
    \end{table}

    \begin{table}[H]
        \centering
        \caption{$S_x$,$S_r$各周波数成分の胴体系伝達後の$F_{y\mathrm{hub}}$}
        \label{table:SxSr_to_Fyhub}
        \setlength{\tabcolsep}{5pt}
        \begin{tabular}{|c|c|c|c|} \hline
            周波数 &  $S_x$荷重成分 & $S_r$荷重成分 &  MRH荷重$F_{y\mathrm{hub}}$ N=4 \\ \hline
            定常 &  $S_{x0}$ &  $S_{r0}$ & $0$ \\ \hline
            \raisebox{-1.2ex}{1/rev} & $S_{x1c}\cos\Psi_m$ & $S_{r1c}\cos\Psi_m$ & $-2S_{r1c}$ \\   \cline{2-4}
                                   & $S_{x1s}\sin\Psi_m$ & $S_{r1s}\sin\Psi_m$ & $-2S_{r1s}$ \\ \hline
            \raisebox{-1.2ex}{2/rev} & $S_{x2c}\cos 2\Psi_m$  & $S_{r2c}\cos 2\Psi_m$  & $0$ \\   \cline{2-4}
                                   & $S_{x2s}\sin 2\Psi_m$  & $S_{r2s}\sin 2\Psi_m$  & $0$ \\ \hline
            \raisebox{-1.2ex}{3/rev} & $S_{x3c}\cos 3\Psi_m$  & $S_{r3c}\cos 3\Psi_m$  & $-2S_{x3c}\cos 4\Psi + 2S_{r3c}\sin 4\Psi$ \\   \cline{2-4}
                                   & $S_{x3s}\sin 3\Psi_m$  & $S_{r3s}\sin 3\Psi_m$  & $-2S_{x3s}\sin 4\Psi - 2S_{r3s}\cos 4\Psi$ \\ \hline
            \raisebox{-1.2ex}{4/rev} & $S_{x4c}\cos 4\Psi_m$  & $S_{r4c}\cos 4\Psi_m$  & $0$ \\   \cline{2-4}
                                   & $S_{x4s}\sin 4\Psi_m$  & $S_{r4s}\sin 4\Psi_m$  & $0$ \\ \hline
            \raisebox{-1.2ex}{5/rev} & $S_{x5c}\cos 5\Psi_m$  & $S_{r5c}\cos 5\Psi_m$  & $-2S_{x5c}\cos 4\Psi - 2S_{r5c}\sin 4\Psi$ \\    \cline{2-4}
                                   & $S_{x5s}\sin 5\Psi_m$  & $S_{r5s}\sin 5\Psi_m$  & $-2S_{x5s}\sin 4\Psi + 2S_{r5s}\cos 4\Psi$ \\ \hline
            \raisebox{-1.2ex}{6/rev} & $S_{x6c}\cos 6\Psi_m$  & $S_{r6c}\cos 6\Psi_m$  & $0$ \\    \cline{2-4}
                                   & $S_{x6s}\sin 6\Psi_m$  & $S_{r6s}\sin 6\Psi_m$  & $0$ \\ \hline
        \end{tabular}
    \end{table}

    \begin{table}[H]
        \centering
        \caption{$N_F$各周波数成分の胴体系伝達後の$M_{x\mathrm{hub}}$}
        \label{table:NF_to_Mxhub}
        \setlength{\tabcolsep}{5pt}
        \begin{tabular}{|c|c|c|} \hline
            周波数 &  $N_F$荷重成分 &   MRH荷重$M_{x\mathrm{hub}}$ N=4 \\ \hline
            定常 &  $N_{F0}$ & $0$ \\ \hline
            \raisebox{-1.2ex}{1/rev} & $N_{F1c}\cos\Psi_m$ & $0$ \\   \cline{2-3}
                                   & $N_{F1s}\sin\Psi_m$ & $2N_{F1s}$ \\ \hline
            \raisebox{-1.2ex}{2/rev} & $N_{F2c}\cos 2\Psi_m$  & $0$ \\   \cline{2-3}
                                   & $N_{F2s}\sin 2\Psi_m$  & $0$ \\ \hline
            \raisebox{-1.2ex}{3/rev} & $N_{F3c}\cos 3\Psi_m$  & $2N_{F3c}\sin 4\psi$ \\   \cline{2-3}
                                   & $N_{F3s}\sin 3\Psi_m$  & $-2N_{F3s}\cos 4\psi$ \\ \hline
            \raisebox{-1.2ex}{4/rev} & $N_{F4c}\cos 4\Psi_m$  & $0$ \\   \cline{2-3}
                                   & $N_{F4s}\sin 4\Psi_m$  & $0$ \\ \hline
            \raisebox{-1.2ex}{5/rev} & $N_{F5c}\cos 5\Psi_m$  & $-2N_{F5c}\sin 4\psi$ \\    \cline{2-3}
                                   & $N_{F5s}\sin 5\Psi_m$  & $2N_{F5s}\cos 4\psi$ \\ \hline
            \raisebox{-1.2ex}{6/rev} & $N_{F6c}\cos 6\Psi_m$  & $0$ \\    \cline{2-3}
                                   & $N_{F6s}\sin 6\Psi_m$  & $0$ \\ \hline
        \end{tabular}
    \end{table}

    \begin{table}[H]
        \centering
        \caption{$N_F$各周波数成分の胴体系伝達後の$M_{y\mathrm{hub}}$}
        \label{table:NF_to_Myhub}
        \setlength{\tabcolsep}{5pt}
        \begin{tabular}{|c|c|c|} \hline
            周波数 &  $N_F$荷重成分 &   MRH荷重$M_{y\mathrm{hub}}$ N=4 \\ \hline
            定常 &  $N_{F0}$ & $0$ \\ \hline
            \raisebox{-1.2ex}{1/rev} & $N_{F1c}\cos\Psi_m$ & $-2N_{F1c}$ \\   \cline{2-3}
                                   & $N_{F1s}\sin\Psi_m$ & $0$ \\ \hline
            \raisebox{-1.2ex}{2/rev} & $N_{F2c}\cos 2\Psi_m$  & $0$ \\   \cline{2-3}
                                   & $N_{F2s}\sin 2\Psi_m$  & $0$ \\ \hline
            \raisebox{-1.2ex}{3/rev} & $N_{F3c}\cos 3\Psi_m$  & $-2N_{F3c}\cos 4\psi$ \\   \cline{2-3}
                                   & $N_{F3s}\sin 3\Psi_m$  & $-2N_{F3s}\sin 4\psi$ \\ \hline
            \raisebox{-1.2ex}{4/rev} & $N_{F4c}\cos 4\Psi_m$  & $0$ \\   \cline{2-3}
                                   & $N_{F4s}\sin 4\Psi_m$  & $0$ \\ \hline
            \raisebox{-1.2ex}{5/rev} & $N_{F5c}\cos 5\Psi_m$  & $-2N_{F5c}\cos 4\psi$ \\    \cline{2-3}
                                   & $N_{F5s}\sin 5\Psi_m$  & $-2N_{F5s}\sin 4\psi$ \\ \hline
            \raisebox{-1.2ex}{6/rev} & $N_{F6c}\cos 6\Psi_m$  & $0$ \\    \cline{2-3}
                                   & $N_{F6s}\sin 6\Psi_m$  & $0$ \\ \hline
        \end{tabular}
    \end{table}

    \begin{table}[H]
        \centering
        \caption{$N_L$各周波数成分の胴体系伝達後の$M_{z\mathrm{hub}}$}
        \label{table:NL_to_Mzhub}
        \setlength{\tabcolsep}{5pt}
        \begin{tabular}{|c|c|c|} \hline
            周波数 &  $N_L$荷重成分 &   MRH荷重$M_{z\mathrm{hub}}$ N=4 \\ \hline
            定常 &  $N_{L0}$ & $4N_{L0}$ \\ \hline
            \raisebox{-1.2ex}{1/rev} & $N_{L1c}\cos\Psi_m$ & $0$ \\   \cline{2-3}
                                   & $N_{L1s}\sin\Psi_m$ & $0$ \\ \hline
            \raisebox{-1.2ex}{2/rev} & $N_{L2c}\cos 2\Psi_m$  & $0$ \\   \cline{2-3}
                                   & $N_{L2s}\sin 2\Psi_m$  & $0$ \\ \hline
            \raisebox{-1.2ex}{3/rev} & $N_{L3c}\cos 3\Psi_m$  & $0$ \\   \cline{2-3}
                                   & $N_{L3s}\sin 3\Psi_m$  & $0$ \\ \hline
            \raisebox{-1.2ex}{4/rev} & $N_{L4c}\cos 4\Psi_m$  & $4N_{L4c}\cos 4\psi$ \\   \cline{2-3}
                                   & $N_{L4s}\sin 4\Psi_m$  & $4N_{L4s}\sin 4\Psi$ \\ \hline
            \raisebox{-1.2ex}{5/rev} & $N_{L5c}\cos 5\Psi_m$  & $0$ \\    \cline{2-3}
                                   & $N_{L5s}\sin 5\Psi_m$  & $0$ \\ \hline
            \raisebox{-1.2ex}{6/rev} & $N_{L6c}\cos 6\Psi_m$  & $0$ \\    \cline{2-3}
                                   & $N_{L6s}\sin 6\Psi_m$  & $0$ \\ \hline
        \end{tabular}
    \end{table}


    \clearpage

    \section{低振動化に対する取り組み}
    \label{sec:vibration_reduction}
    \par
    回転翼機が高速で前進巡航飛行する場合,メインロータハブに作用する振動荷重およびモーメントが増大することは,前項で述べたとおりである.
    この振動の増大は,プライマリサーボよりメインロータ側に存在するすべてのメインロータダイナミックコンポーネントの構成品における疲労寿命の低下を招く.
    さらに,振動の下流に位置する胴体構造に対しても疲労強度の低下を引き起こす要因となる.
    また,これらの振動荷重およびモーメントが胴体を加振することにより,パイロット座席において発生する加速度が増大し,
    パイロット人体の疲労や不快感を生起することとなる.
    \supcite{stupar_2012,farrar_2007,ali_2010,Jovanovic2010,Jovanovic2011,Harris1996}
    さらに,エンジン,装備品,計器類,航法支援装置,武装システムなどの機器に対しても,環境振動に関する機能上の許容値を超過するおそれが生じる.
    低振動化への対策を講じない場合,上述したダイナミックコンポーネントおよび胴体構造の疲労強度低下に対しては,
    部品の肉厚増加などの設計変更によって対応せざるを得ず,結果として機体重量の増加を招く.
    また,人体の疲労増大はミッションパフォーマンスの低下につながり,装備品の誤作動や機能停止は飛行安全そのものを脅かす要因となる.
    近年,MILスタンダードにおける振動要求値は,改訂が行われるたびに段階的に下がっており,より低振動な機体に対する要求が高まっている.
    さらに,今後はコンパウンドヘリコプタに代表される高速飛行可能な回転翼機の需要が一層高まることが予想されている.\supcite{Ferguson2017}
    これらの背景から,高速巡航飛行時における振動低減技術は,今後ますます重要な技術課題となることが予想される.

    \par
    胴体の振動を低減する手法は,大きく分けて二つの様式に分類される.
    すなわち,胴体に代表される固定系における対策と,メインロータブレードに代表される回転系における対策である.
    前者の固定系における対策としては,ダイナミックバイブレーションアブソーバー(Dynamic Vibration Absorber,DVA)や,
    アクティブバイブレーションコントロール(Active Vibration Control,AVC)が代表的な手法として挙げられる.
    Fig.\ref{fig:DVA_example},Fig.\ref{fig:AVC_example}にこれらの概念を示す.
    一般に,DVAが十分な制振性能を発揮するためには,機体質量のおよそ2倍に相当する動マスを有するDVAを搭載する必要があるとされており,
    航空機に搭載する装備品としては重量面の観点から必ずしも好ましいものではない.
    また,固定系における対策としては,いわゆる“nodamatic system”と呼ばれる振動絶縁装置も知られている.
    Fig.\ref{fig:nodynamic_system}に示すように,MGBの胴体取付部に梁の曲げ変形を利用して荷重を伝達する機構を設け,
    MGB取付部がモード形状における節の位置となるよう,あらかじめ設計する手法である.
    Fig.\ref{fig:DAVI}には,Dynamic Antiresonant Vibration Isolator(DAVI)の装置写真を示す. 
    DAVIは“nodamatic system”と同様に,MGBと胴体を接続する荷重伝達部において,バネおよびウェイトから構成される機構を用い,
    固有振動数が応答の小さくなる反共振点となるよう設計することで,振動の伝達を低減することを目的とした装置である.

    \begin{figure}[htbp]
    \centering
        % 1行目
        \begin{subfigure}[b]{0.45\linewidth}
            \centering
            \includegraphics[width=\linewidth]{photo/DVA.png}
            %\caption{DVA}
            \label{fig:DVA}
        \end{subfigure}
        \hfill
        \begin{subfigure}[b]{0.45\linewidth}
            \centering
            \includegraphics[width=\linewidth]{photo/DVA_frequency.png}
            %\caption{周波数と振幅}
            \label{fig:DVA_frequency}
        \end{subfigure}
    \caption{DVA模式図\supcite{Bielawa2006}}
    \label{fig:DVA_example}
    \end{figure}


    \begin{figure}[htbp]
    \centering
        % 1行目
        \begin{subfigure}[b]{0.7\linewidth}
            \centering
            \includegraphics[width=\linewidth]{photo/AVC.png}
            \caption{AVCの概要}
            \label{fig:AVC}
        \end{subfigure}
        
        \vspace{5mm}

        \begin{subfigure}[b]{0.7\linewidth}
            \centering
            \includegraphics[width=\linewidth]{photo/AVC_equipment.png}
            \caption{向かって左から右の順に:動力発生用アクチュエータ,アクティブ制振および電気ユニット,アクティブ制御センサ,ヒューマンマシンインターフェース}
            \label{fig:AVC_equipment}
        \end{subfigure}
    \caption{AVCの代表例\supcite{AVC}}
    \label{fig:AVC_example}
    \end{figure}


    \begin{figure}[H]
      \centering
      \includegraphics[keepaspectratio, width=0.6\linewidth]{photo/nodinamic_system.png}
      \caption{nodamatic system 概念図\supcite{gaffey_1976}}
      \label{fig:nodynamic_system}
    \end{figure}    


    \begin{figure}[H]
      \centering
      \includegraphics[keepaspectratio, width=0.6\linewidth]{photo/DAVI.png}
      \caption{DAVI\supcite{flannelly_1966,schuett_1969,rita_1978}}
      \label{fig:DAVI}
    \end{figure} 


    \par
    一方,回転系における対策としては,パッシブな手法としてメインロータブレードのルート部にペンデュラム(遠心振り子)を装着する方法や,
    メインロータハブ中央位置にバイファイラを装着する方法が一般的である(Fig.\ref{fig:pendulum_bifilar}, Fig.\ref{fig:pendulum_bifilar_system})
    ペンデュラムはTable.\ref{table:Sz_to_Fzhub}に示した$F_z$成分を低減することを目的として,$N$/revに共振点を設定する.
    一方,バイファイラはTable.\ref{table:SxSr_to_Fxhub},Table.\ref{table:SxSr_to_Fyhub}に示した
    $F_{x\mathrm{hub}}$,$F_{y\mathrm{hub}}$成分を低減するため,$(N-1)$/rev あるいは$(N+1)$/revに合わせて共振点を調整することが基本である.
    このうちバイファイラについては,遠心振り子のような完全なパッシブ装置にとどまらず,アクティブな装置についても試作品が開発されており,
    パッシブ型バイファイラと比較して大幅な重量軽減が実現されている.
    また,飛行試験においても一定の効果が確認されているものの,実用化には至っていない.(HMVS: Hub Mounting Vibration Suppression\supcite{kakaley_2018}).
    これまでに述べた回転系および固定系の振動低減手法はいずれも,十分な振動低減効果を得るためには相応の質量を必要とし,航空機への搭載という観点からは課題となる.
    さらに,これらの装置は調整が極めてセンシティブであり,メインロータ回転数のわずかな変動によっても振動低減効果に有意な差が生じるため,
    調整作業に多大な労力を要する.
    また,振動低減のために増加した質量を抑制する目的でアクティブ化などの対策を講じる必要が生じ,結果としてコスト増を招き,運用者側に対する負担が大きくなる.
    
    
    \begin{figure}[H]
      \centering
      \captionsetup{justification=centering}
      \includegraphics[keepaspectratio, width=0.8\linewidth]{photo/pendulum_bifilar.png}
      \caption[ペンデュラムおよびバイファイラ]{
              ペンデュラムおよびバイファイラ\supcite{johnson_2013} \\
               \quad (a)ペンデュラムの事例 (b)バイファイラの事例 \\
               各写真の下部に機種名を示す.
              }
      \label{fig:pendulum_bifilar}
    \end{figure} 
    
    \clearpage
    \begin{figure}[htbp]
    \centering
        % 1行目
        \begin{subfigure}[b]{0.45\linewidth}
            \centering
            \includegraphics[width=\linewidth]{photo/pendulum.png}
            \caption{Blade pendulum absorbers}
            \label{fig:pendulum}
        \end{subfigure}
        \hfill
        \begin{subfigure}[b]{0.45\linewidth}
            \centering
            \includegraphics[width=\linewidth]{photo/bifilar.png}
            \caption{bifilar vibration absorber for rotor hub}
            \label{fig:bifilar}
        \end{subfigure}
    \caption{ペンデュラムおよびバイファイラ模式図\supcite{johnson_2013}}
    \label{fig:pendulum_bifilar_system}
    \end{figure}     
    

    \par
    回転系における対策には,もう一つの代表的な手法としてTEF(Trailing Edge Flaps)が挙げられる.
    Fig.\ref{fig:TEF}にその概念図を示す.
    TEFの初期のアイディアはStraub and Charles\supcite{straub_1990}により紹介され,その後Ormiston\supcite{ormiston_2001}によって発展が図られた.
    TEFは事前の数値解析および実証試験において非常に大きな労力を要する手法であり,当初の目標であったスワッシュプレートレス設計を実現するためには,
    TEFを駆動するスマート材料において十分な変位量を確保することが困難であることが指摘されている.
    \supcite{shen_chopra_2002a,shen_chopra_2002b,shen_chopra_2004a,shen_chopra_2006,bluman_gandhi_2011} 
    さらに, 回転系のTEFに対して,胴体側の固定系からスリップリングを介して十分な電力を供給する点においても,技術的な困難を伴うことが明らかとなっている.
    しかしながら,TEFはこれらの技術的課題が未解決であるにもかかわらず,プライマリコントロールのみならず振動低減を目的としたデバイスとして極めて有望な手法であった.
    4 枚ブレードを有する回転翼機においては,メインロータハブ中央位置の$x_{\mathrm{hub}}$,$y_{\mathrm{hub}}$,$z_{\mathrm{hub}}$座標系における
    $F_{z\mathrm{hub}}$,および$M_{x\mathrm{hub}}$,$M_{y\mathrm{hub}}$の4/rev振動荷重およびモーメントが,約90\%低減可能であることが解析で示されている.
    \supcite{shen_chopra_2004b,friedmann_millott_1995}
    一方で, 実機においては前述のとおり,TEFを駆動するアクチュエータの変位量が不足しており,現時点ではこの水準の振動低減効果を実現することは困難である.
    さらに,空力弾性学的な観点からも慎重な検討が必要であることが明らかとなっている.\supcite{shen_chopra_2003}

    \begin{figure}[H]
      \centering
      \includegraphics[keepaspectratio, width=0.6\linewidth]{photo/TEF.png}
      \caption{TEF(Trailing Edge Flaps)\supcite{friedmann_2004}}
      \label{fig:TEF}
    \end{figure} 


    \par    
    ここまでに述べたとおり,回転翼機の振動を低減するためには,一定の質量増加を受け入れる必要があり,またその調整には多くの時間と労力を要する.
    さらに,一部の手法については革新的な技術の進展を待たなければならない課題も存在する.
    では,生来振動の少ない機体を実現するためには,どのような方策が考えられるであろうか.
    先行研究として,Gandhi,F.S. and Sekura,M.K.\supcite{gandhi_sekura_2004}は,回転翼機後方に配置された水平尾翼の舵角を変更することにより,
    メインロータハブ中央位置に定常的なモーメント($M_x$,$M_y$)および$F_z$を付与した場合の振動変化を解析的に検討している.
    同研究では,UH-60およびBo-105の2種類の機体を対象とし,UMARCを用いた解析により,メインロータハブ中央位置に作用する4/rev振動荷重およびモーメントが評価された.
    評価指標としては,$F_{x\mathrm{hub}}$,$F_{y\mathrm{hub}}$,$F_{z\mathrm{hub}}$,および$M_{x\mathrm{hub}}$,$M_{y\mathrm{hub}}$,$M_{z\mathrm{hub}}$
    の6成分を無次元化した値の二乗平均値$J$(Vibration Index)が用いられ,この指標に基づいて結果が整理されている.
    その結果,ベースラインのスタビレータ舵角設定とは異なる舵角において,振動が極小値を示すことが明らかとなった.
    ただし,Ref.\cite{gandhi_sekura_2004}はにおける検討は高速巡航時に限定されている.
    一方で,中速域においても振動レベルは必ずしも小さくないことが知られており,さらに中速域ではスタビレータに作用する空力荷重が低下するため,
    スタビレータ舵角の変更による振動低減効果が十分に得られない可能性があると考えられる.


    \clearpage    
    \section{本論文の目的}
    \label{sec:thesis_perpose}
    \par
    本論文では,機体の重心移動による振動低減の可能性について検討を行う.
    既に小型から中型の回転翼機においては,サイクリックピッチの代替手法として,重心移動による機体制御が可能であることが解析的に示されている.
    (\supcite{yoshizaki_2006,steiner_2008})
    本研究では,メインロータハブ中央位置に作用する振動荷重およびモーメントを低減することを目的として,機体の重心位置を前後および左右方向に
    移動させることにより,機体座標系における定常的な$M_x$,$M_y$をメインロータハブ中央位置に作用させる手法について検討する.
    重心移動は,機体内部に搭載された装備品を移動させることによって実現するものとし,可動式の重心移動機構は想定しない.
    重心移動により作用する$M_x$,$M_y$は,機体速度に依存せず,高速域以外の飛行条件においても振動低減効果が期待できる.
    第\ref{chap:flight_test}章において,ラジコンヘリコプタを対象として飛行試験を行い,重心移動による低振動化の効果を確認する.
    第\ref{chap:camrad_analysis}章では,CAMRAD $\mathrm{II}$という回転翼機の空力解析ツールを用いて,トリム解析を実施する.
    得られた解析結果に基づき,重心移動が振動荷重およびモーメントに与える影響について考察するとともに,解析結果の妥当性について論じる.
    CAMRAD $\mathrm{II}$によるトリム解析では,各重心位置において機体重心点に作用する荷重およびモーメントの釣合いが解かれるが,
    同時にメインロータハブ中央位置に作用する空力起因の振動荷重およびモーメントも導出される.
    また,振動低減が生じる要因についても検討する.
    本研究では重心移動に可動方式を採用しないことから,重心移動に伴って変化するトリム状態が機体に与える影響について検討する必要がある.
    そこで第\ref{sec:power}節では,重心移動がメインロータの必要パワーに与える影響について検討する.
    さらに,第\ref{chap:leadlag}章では,ラジコンヘリコプタはフラッピングヒンジが存在しないため,リード・ラグヒンジの有無をパラメータとして解析を行い,
    $F_{z\mathrm{hub}}$の4/rev成分が変化する原因について考察する.

    \clearpage
    \section{使用する解析ツール}
    \label{sec:analysis_method}
    \par
    本論文では,回転翼機の振動と機体重心位置との関係性について,解析的な検討を実施する.
    本解析により,機体のトリム姿勢角,メインロータハブ中央位置に作用する振動荷重およびモーメント,
    ならびにメインロータブレードが1回転する間の各アジマス角における諸量が得られる.
    これまでに,回転翼機の包括的解析ツールは産官学において数多く開発されてきた.
    Fig.\ref{fig:CAMRADhistory}には,代表的な解析ツールとその開発者および開発時期を示す.
    本論文で使用する解析ツールは,回転翼機の振動解析において豊富な実績を有するCAMRAD $\mathrm{II}$\supcite{johnson_1998}である.
    CAMRAD $\mathrm{II}$は,シングルロータヘリコプタに限らず,多様な形態の回転翼機を対象とした包括的な空力解析ツールである.
    本ツールは,NASAに所属していたWayne Johnsonにより,NASA および米陸軍向けに開発されたCAMRADを起源としている.
    CAMRADは1978年から1979年にかけて開発され,当時は回転翼機を包括的に解析する技術が存在せず,個別の用途に応じた解析が行われていた.
    初期のCAMRADでは,メインロータおよびテールロータから構成されるシングルロータヘリコプタを対象とし,
    各ブレードを梁要素として取り扱うとともに,Scully Vortex Wake Modelに基づくインフロー計算が用いられていた.
    その後,1986年から1989年にかけてJohnson AeronauticsによりCAMRAD/JAとして改良が加えられ,特にインフロー計算の精度が大幅に向上した.
    この過程で開発された自由後流法(Free Wake Model)は,COPTER,UMARC,2GCHASなどの同種の解析ソフトウェアにも採用されている.
    回転翼機の解析は,空気力学,構造力学,振動工学など複数の力学分野を同時に考慮する必要があり,極めて複雑である.
    こうした背景のもと,CAMRADは最新版であるCAMRAD $\mathrm{II}$へと発展し,
    メインロータブレードを含む胴体やドライブシャフトを弾性体として取り扱うマルチボディダイナミクスと,回転翼の空気力学を統合した解析が可能となった.
    CAMRAD $\mathrm{II}$による解析は,トリム解析,非定常解析,およびフラッター解析の三つの主要なタスクから構成されている.
    本ツールは,メインロータハブ中央位置における振動荷重およびモーメントを高い精度で解析できることが示されており\supcite{Lim2003},
    本論文で評価対象とする過渡応答解析および安定性解析も実施可能である.
    本論文では,トリム解析においてメインロータ回転面を通過するインフローを求める際に,自由後流モデル(Free Wake Model)を用いた.

    \begin{figure}[H]
      \centering
      \includegraphics[keepaspectratio, width=0.8\linewidth]{photo/CAMRADhistory.png}
      \caption{回転翼機の包括的解析ツール\supcite{johnson_1998}}
      \label{fig:CAMRADhistory}
    \end{figure}
