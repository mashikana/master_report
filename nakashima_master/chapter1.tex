\chapter{序論}
\begin{flushright}
	\begin{minipage}{0.8\hsize}
		\quad 本章では,本論文の研究背景と研究目的について説明する.
		まず,マルチロータ機の現状と運用上の課題である外乱に対する安定性の低さと制御の応答の重要性について実例をもとに説明する.
    次に,外乱に対する制御の応答の遅さを解決するための対策手法として可変ピッチロータ搭載型マルチロータ機を検討し,その課題を示す.
    そして,本研究の目的とそれによって達成される利点を述べる.
    最後に本研究の具体的な流れについて説明する.
	\end{minipage}
\end{flushright}
    \section{研究背景}
    \par
    回転翼機は固定翼機と異なりホバリングや上昇・下降といった鉛直方向の移動が可能であることから,
    災害救助活動やドクターヘリなどによる患者輸送活動などに活用されている. 
    固定翼機が主に主翼により発生する揚力を回転翼機では一般に一定の角速度で回転するブレードと呼ばれる翼により発生する.
    通常2本以上の複数のブレードにより構成されるメインロータが機体の運動をコントロールすることになる.
    メインロータブレードはスワッシュプレートと呼ばれる機構を介してピッチ角を定常的に或いは周期的に変化させることが可能であり,
    このことにより機体の上昇,降下,前進横進といったコントロールが可能となる.
    この一方でメインロータは常にその回転方向と逆回転方向にトルクを発生させる.
    このトルクを打ち消すために,回転翼機にはさまざまな様式がある.
    まずは,SH-60Kに代表されるメインロータと機体尾部にトルクを打ち消す側方荷重を発生させるテールロータからなるコンベンショナル
    なシングルロータヘリコプタ1が挙げられる. 
    また,CH-47に代表されるタンデムロータヘリコプタ2に関しては機体前後軸に配置された2つのロータが各々胴体に作用するトルクを
    打ち消しあうことで機体を安定させることが出来る. 
    また,ロータが機体前後方向に存在することから胴体の重心移動に対してロバストである.
    メインロータを上下方向に2段構成として逆方向に回転させテールロータを不要としたKa-32などに代表される二重反転式ロータヘリコプタ,
    さらには二重反転式ロータの機構上の複雑さを簡素化して整備性を向上したK-MAXに代表される交差反転式ロータヘリコプタ4などが挙げられる(図1-1)
    どの様式においても,前進飛行する場合においてロータ面は飛行速度ベクトルに対してほぼ平行となるエッジワイズフライトとなる.
    この状態が固定翼機におけるプロペラとは大きく異なり回転翼まわりの空気の流れに非軸対称性をもたらしており,振動の発生要因となる.
    このとき,例えばシングルロータの上面図を(図1-2)に示すとメインロータブレードに作用する空力荷重はメインロータブレードの位置を表す.
    アジマス角$\Psi$に応じて異なり,前進側ブレード($\Psi$ =90 deg)と後退側ブレード($\Psi$ =270 deg)で
    図中に示される式で表される速度から求められる.    

    \par
    高速で前進飛行する場合において,N/rev振動(N:ブレード枚数)が卓越する結果となる.
    このN/rev振動は メインロータブレードに働くメインロータ回転面外方向の荷重およびモーメントのうち
    N/rev成分のみが機体に対して成分のみが機体に対してN/rev成分として伝達すること成分として伝達すること,
    またメインロータ回転面内方向のメインロータ回転面内方向の(N-1)/rev,ならびに(N+1)/rev荷重およびモーメントが
    N/rev荷重およびモーメントとして機体に伝達することに由来するものである.
    メインメインロータブレードルート部においてロータブレードルート部において1/rev以上の高調波成分以上の高調波成分の荷重
    およびモーメントの荷重およびモーメントはは以以下の理由により発生する下の理由により発生する.
    回転翼機が回転翼機が前進飛行する際に前進飛行する際に,メインロータブレード回転面へのインフロー分布が,
    回転するブレードブレード翼素の翼素の$\alpha$に影響を与えに影響を与える.
    結果として発生するルート部の荷重およびモーメントはるルート部の荷重およびモーメントは,
    メインロータブレードが弾性体であるがゆえに,N/rev成分成分((N=1,2,……))近傍に存在する固有振動モード近傍に存在する
    固有振動モードが励起するが励起することにより1/rev以上以上の高調波成分を含むこととなる.
    通常,初度の開発設計の中でその影響を小さくするよう設計されているものの発生自体を抑えることは困難である.
    特に面外方向の振動荷重は回転する各々1本に発生するN/rev荷重が胴体側にN/rev荷重として伝達する.
    このため,高調波成分も主として1/rev成分として見える荷重に含まれる高調波成分によるものとなる.
    N枚のメインロータブレードが厳密に管理された工程により形状,重量剛性のいずれも極めて均一な品質を保有していると仮定するならば,
    それ以外の高調波成分は機体座標系に伝達する際に各々キャンセルされ,結果として極めて小さな値となり,問題とはならない.
    ここで N/rev振動 は メインロータハブ中央位置における $x_{\mathrm{hub}}$,$y_{\mathrm{hub}}$,$z_{\mathrm{hub}}$座標系で定義されるN/rev成分の
    $F_{x\mathrm{hub}}$,$F_{y\mathrm{hub}}$,$F_{z\mathrm{hub}}$,$M_{x\mathrm{hub}}$,$M_{y\mathrm{hub}}$,$M_{z\mathrm{hub}}$から生ずるものであり
    これらのN/rev成分は,その荷重あるいはモーメントがインプレーンかアウトオブプレーンかによってメインロータ回転系における振動成分が異なる.(表 1-1~表 1-6)
    回転翼機の振動は他にもシングルロータにおけるテールロータからの振動やエンジンや駆動系統からの振動,さらには操縦系統からのトランジェントな入力,
    ガスト応答などがあるもののメインロータ由来のN/rev振動と比較して十分に小さいものであるため本論文では議論の対象としない.

    \clearpage 

    \par
    表(1-1)~表(1-6)に示すメインロータハブ中央位置における荷重およびモーメントは回転するブレードが空力荷重を受けて
    結果的に固定側である胴体側に伝わる荷重をまとめており,回転系でのブレードルート部における荷重の説明を(図1-3)に示す.

    \clearpage

    \par
    ここで(図1-3)に示すように,$N$本ブレードのヘリコプタにおいて,$m$番目(アジマス角 $\Psi_m$に位置する)のブレードの
    ルート部における荷重は一般に定常成分と1/rev成分およびその高調波成分より構成される.
    例えば,$S_z$については,周波数ごとにその成分を記載することにより,(表 1-1)の左列の総和となる
    (6/rev以上は省略する.また (表1-1)は$F_{z\mathrm{hub}}$について示しており,メインロータ回転面内の荷重である$F_{x\mathrm{hub}}$,$F_{y\mathrm{hub}}$については
    (表1-2)及び(表1-3)に,また,$M_{x\mathrm{hub}}$,$M_{y\mathrm{hub}}$および,$M_{z\mathrm{hub}}$については(表 1-4~表 1-6)に示す.
    (表1-1)~(表1-6)は 左列に周波数ごとにまとめられている.
    回転しているブレードルート荷重が回転をしていないMRH座標系にどのように伝達するかをまとめている.
    これらは全てブレード枚数 N=4の結果であり,胴体系伝達後の荷重およびモーメントは(表1-1),(表1-6)の結果から分かるようにアウトオブプレーンの
    荷重$F_{z\mathrm{hub}}$および モーメント$M_{z\mathrm{hub}}$は 定常成分および4/rev成分が周波数の変調を起こすことなく,回転系から胴体系に伝わる.
    また,4/rev以外の高調波成分は0となる一方で,インプレーンの荷重およびモーメントは(表 1-2~表 1-5)から分かるように,
    回転系の3/rev,5/revの成分が変調して,胴体系に4/revとして伝達する.
    また,1/rev成分が定常成分として伝達し,それ以外の高調波成分は0となる.(なお,伝達後のアジマス角$\psi$は$m=1$のアジマス角である)
    それぞれの荷重モーメント成分の周波数の変調については文献10に示されている通りであるが
    特に変調後にキャンセルされて0とならないケースについては補遺Aに示す




    \section{低振動化に対する取り組み}
    \par
    回転翼機が高速で前進巡航飛行しているとき,メインロータハブ に働く振動荷重及びモーメントが大きくなることは前項で述べたとおりである.
    このことは プライマリサーボよりメインロータ側に存在する全てのメインロータダイナミックコンポーネントの構成品の疲労寿命の減少,
    更に振動の下流である胴体構造の疲労強度の低下に繋がる.
    また,その振動荷重及びモーメントが胴体を加振することにより生ずるパイロット座席での加速度がパイロット人体の疲労や不快感を生起することとなる(11-16)
    さらには,エンジン,装備品,計器類,航法支援装置や武装システムといった機器の機能上の環境振動における許容値を超えてしまうおそれも生じる.
    低振動化に取り組まなければ,上述のダイナミックコンポーネントや胴体構造の疲労強度の低下に対しては部品の肉厚の増加などで対応せざるを得ないため,
    重量増加を招くこと人体の疲労に関してはミッションのパフォーマンス低下につながること,装備品の誤作動や機能停止に関しては
    飛行安全そのものを脅かすこととなってしまう.
    近年,MILスタンダードの振動要求値は改訂版が発行される度に段階的に下がっており,より低振動に対する要求が高まっていることや,
    更には,今後コンパウンドヘリコプタに代表される高速で飛行できる回転翼機の需要が高まっていくことが予想されるため(17),
    高速巡航時の振動低減技術はより一層重要なものとなっていくことが予想される.

    \par
    胴体の振動を低下させる方法には大きく2種類の様式がある.
    それは,胴体に代表される固定系での対策とメインロータブレードを代表とする回転系における対策の2つである.
    前者については代表的なものとしてダイナミックバイブレーションアブソーバー(DVA やアクティブバイブレーションコントロール(AVCが挙げられる (18)図1-4
    一般に,DVAが十分な制振能力を発揮するためには機体質量の約2倍の動マスを保有したDVAを搭載する必要があるといわれており,航空機に搭載する装備品としては好ましくない.
    また 固定系での対策としては他にいわゆる"nodamatic system"(19)といわれる振動絶縁装置がある.図1-5に示すように MGBの胴体取付部に梁の曲げで荷重伝達をする機構を設けて
    MGB取付部においてモード形の節の位置になるよう,あらかじめ設計しておく方法である.
    図 1-6には Dynamic Antiresonant Vibration Isolator(DAVI)(20)-(22)の 装置写真 を示す.
    DAVIは ”nodamatic system”と同様に,MGBと胴体をつなぐ荷重伝達部においてバネとウェイトから構成されており 固有振動数を応答が小さくなる反共振点になるよう設計し,
    振動の伝達を小さくするよう考えられたものである.類似の例として The Improved Rotor Isolation System (IRIS)(23)The hydraulic antiresonant isolator(24)
    さらにはThe Liquid Inertia Vibration Elimination (LIVE)(25)などが挙げられるが,ここでは詳細は割愛する.

    \par
    一方,後者に関してはパッシブな手法としてメインロータブレードのルート部にペンデュラム(遠心振り子)やメインロータハブ 中央位置にバイファイラを装着する手法が一般的である.(図 1-7 図 1-8)
    ペンデュラムは 図 1-1で示した $F_z$成分を低減するため,N/revに またバイファイラは同じく図 1-1で示した $F_{x\mathrm{hub}}$,$F_{y\mathrm{hub}}$成分を低減するため(N-1)/rev,
    あるいは (N+1)/revに合わせて共振点を調整することが基本的である.

    \par
    このうちバイファイラに関しては遠心振り子のようなパッシブな装置ではなく,アクティブな装置についても試作品が完成しており,パッシブなバイファイラよりも重量軽減することが実現している.
    また,飛行試験でも効果が確認されているが実用には至っていない.(HMVS: Hub Mounting Vibration Suppression)
    ここまで述べた振動低減手法についてはいずれも振動低減の効果を十分得るためにはそれなりの質量が必要であり航空機に搭載するうえで不利益となる.
    さらに,調整が非常にセンシティブであり,メインロータ回転数の僅かな変動でもその振動低減効果に有意差が生じるため,その調整に非常に多くの労力を払う必要がある.
    また,振動低減のために増加した質量を削減するためにはアクティブ化等の努力が必要となり,結果としてコスト増を招くため,運用者側に対する負担が大きいものとなる.


    \par
    回転系における対策には代表的なものがもう1種類あり,それはTEF(Trailing Edge Flaps)と呼ばれる手法であり,図1-9にその概念図を示す.
    これは初期のアイディアはStraub and Charles(28)により紹介され Ormiston(29)がそれに続いた.
    TEFは事前の数値解析やその実証試験において非常に大きな努力が必要であり,当初の目標であったスワッシュプレートレス設計を実現させるためにはTEFを
    駆動するスマート材料では十分な変位量を確保することが難しい(30ー34)こと,また回転系のTEFに胴体側固定系からスリップリングを介して十分な電力を供給することにも技術的な困難を伴うことが生じた.
    しかしながら,TEFはその技術的課題を克服することを待たなくてもプライマリコントロールのみならず振動低減のためのデバイスとしてとても有望なものであった.
    4枚ブレードの回転翼機においてメインロータハブ中央位置での$x_{\mathrm{hub}}$,$y_{\mathrm{hub}}$,$z_{\mathrm{hub}}$座標系における$F_{z\mathrm{hub}}$や
    $M_{x\mathrm{hub}}$,$M_{y\mathrm{hub}}$の 4/rev振動荷重及びモーメントが90\%低減することが解析的に求められた.(35-36)
    しかし実際には上述の通り,TEFを駆動するアクチュエータの変位量が不足しておりここまでの振動低減は今のところ望めない.
    そのうえ,空力弾性学上の問題に関して慎重な検討が必要であることが分かった.(37)

    \par
    ここまで述べたように回転翼機の振動を低減させるためにはそれなりの質量増加を覚悟する必要があること,またその調整に多くの時間を費やし,革新的な技術の進展を待たねばならない項目もある.
    生来振動の少ない機体とするにはどうすればよいのだろうか.

    \par
    先行する研究としてGandhi, F. S. and Sekura, M. K. (38)は 回転翼機の後方に位置する水平尾翼の舵角を変更してメインロータハブ中央位置に
    定常的なモーメント($M_x$,$M_y$)及び$F_z$を 付与した際の振動の変化を解析的に検討した.
    UH-60及びBo-105の2種類の機体において,UMARCを用いた解析を実施してメインロータ中央位置に働く4/rev振動荷重及びモーメントを検討した.
    6つの成分$F_{x\mathrm{hub}}$,$F_{y\mathrm{hub}}$,$F_{z\mathrm{hub}}$及び$M_{x\mathrm{hub}}$,$M_{y\mathrm{hub}}$,$M_{z\mathrm{hub}}$の無次元化値の
    2乗平均値 J(Vibration Index)が評価指標とされ,その値で結果が整理された.
    その結果,ベースラインのスタビレータ舵角セッティングとは異なる角度で振動が極小値をとることが判明した.
    論文 (38)では高速巡航時での検討にとどまるが,中速域においても振動は小さくないことが分かっており,さらに中速領域ではスタビレータに作用する空力荷重が小さくなるため,
    スタビレータ舵角の変更による低振動化には十分な効果が得られない可能性が考えられた



    \begin{figure}[H]
      \centering
      \includegraphics[keepaspectratio, width=0.6\linewidth]{photo/module.png}
      \caption[Module for controlling collective pitch]{Module for controlling collective pitch}
      \label{fig:sarbo}
    \end{figure}

    \begin{figure}[H]
      \centering
      \includegraphics[keepaspectratio, width=0.6\linewidth]{photo/collective_drone.jpg}
      \caption[Variable pitch multirotor aircraft AXM-1000]{Variable pitch multirotor aircraft AXM-1000 }
      \label{fig:collective}
    \end{figure}
    

    \section{本論文の目的}
    \par
    本論文では重心移動による振動低減の可能性について検討を実施した.
    すでに,小型~中型機においてはサイクリックピッチの代替手法として重心移動により機体のコントロールが可能であることが解析的に検討されている (39,40
本論文では
まずは第 2章において メインロータハブ 中央位置に作用する振動荷重
及びモーメントを低減させるために機体の重心位置を前後左右に移動させることで メ
インロータハブ 中央位置に 機体座標系における 定常的な 𝑀𝑥 𝑀𝑦 を作用させ ることを
検討した 重心移動は機体内部の装備品を移動させることで実現させることを考えてお
り 可動式としては考えない 重心移動により付加することができる 𝑀𝑥 𝑀𝑦は機体速
度によらず高速域以外でも振動低減できることが期待できる 本解析のために使用する
ツールは CAMRAD IIとし このツールでトリム解析を実施した結果について考察を加
え 解析結果の妥当性について論じる CAMRAD IIではこのトリム解析においては様々
な重心位置において機体重心位置における荷重およびモーメントのバランスを解いて
いるが この時のメインロータハブ中央位置に作用する空力による振動荷重およびモー
メントも導出される 第 3章では第 2章で得られた 振動荷重及びモーメントを 示す ま
た パイロット座席位置 を機体振動の代表的な評価点として考え その位置での 加速度
を 解析的に示す また CAMRAD IIにて得られたメインロータハブ中央位置での振動
荷重およびモーメントを使用してパイロット座席位置における加速度応答を数値解析
で 求め 考察を加える また 振動低減が起きる理由についても考察する また 機体
の重心移動に関しては可動方式を採らないことから 重心移動により変化するトリムが
与える影響を 検討する 必要がある 第 4章では重心移動が メインロータの必要パワー に
与える影響 と機体の固有 安定性 に与える影響を検討する
第
5章では振動低減のために必要な 重心移動が 機体内部の装備品の移動だけでは
対応できない状況も想定して重心移動の代替手法としてメインロータシャフトの傾き
(前後 左右方向)の効果についても検討を実施する