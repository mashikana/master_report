\chapter{序論}
\begin{flushright}
	\begin{minipage}{0.8\hsize}
		\quad 本章では,本論文の研究背景と研究目的について説明する.
		まず,マルチロータ機の現状と運用上の課題である外乱に対する安定性の低さと制御の応答の重要性について実例をもとに説明する.
    次に,外乱に対する制御の応答の遅さを解決するための対策手法として可変ピッチロータ搭載型マルチロータ機を検討し,その課題を示す.
    そして,本研究の目的とそれによって達成される利点を述べる.
    最後に本研究の具体的な流れについて説明する.
	\end{minipage}
\end{flushright}
    \section{研究背景}
    \par
    回転翼機は固定翼機と異なりホバリングや上昇・下降といった鉛直方向の移動が可能であることから,
    災害救助活動やドクターヘリなどによる患者輸送活動などに活用されている. 
    固定翼機が主に主翼により発生する揚力を回転翼機では一般に一定の角速度で回転するブレードと呼ばれる翼により発生する.
    通常2本以上の複数のブレードにより構成されるメインロータが機体の運動をコントロールすることになる.
    メインロータブレードはスワッシュプレートと呼ばれる機構を介してピッチ角を定常的に或いは周期的に変化させることが可能であり,
    このことにより機体の上昇,降下,前進横進といったコントロールが可能となる.
    この一方でメインロータは常にその回転方向と逆回転方向にトルクを発生させる.
    このトルクを打ち消すために,回転翼機にはさまざまな様式がある.
    まずは,SH-60Kに代表されるメインロータと機体尾部にトルクを打ち消す側方荷重を発生させるテールロータからなるコンベンショナル
    なシングルロータヘリコプタ1が挙げられる. 
    また,CH-47に代表されるタンデムロータヘリコプタ2に関しては機体前後軸に配置された2つのロータが各々胴体に作用するトルクを
    打ち消しあうことで機体を安定させることが出来る. 
    また,ロータが機体前後方向に存在することから胴体の重心移動に対してロバストである.
    メインロータを上下方向に2段構成として逆方向に回転させテールロータを不要としたKa-32などに代表される二重反転式ロータヘリコプタ,
    さらには二重反転式ロータの機構上の複雑さを簡素化して整備性を向上したK-MAXに代表される交差反転式ロータヘリコプタ4などが挙げられる(図1-1)
    どの様式においても,前進飛行する場合においてロータ面は飛行速度ベクトルに対してほぼ平行となるエッジワイズフライトとなる.
    この状態が固定翼機におけるプロペラとは大きく異なり回転翼まわりの空気の流れに非軸対称性をもたらしており,振動の発生要因となる.
    このとき,例えばシングルロータの上面図を(図1-2)に示すとメインロータブレードに作用する空力荷重はメインロータブレードの位置を表す.
    アジマス角$\Psi$に応じて異なり,前進側ブレード($\Psi$ =90 deg)と後退側ブレード($\Psi$ =270 deg)で
    図中に示される式で表される速度から求められる.    

    \par
    高速で前進飛行する場合において,N/rev振動(N:ブレード枚数)が卓越する結果となる.
    このN/rev振動は メインロータブレードに働くメインロータ回転面外方向の荷重およびモーメントのうち
    N/rev成分のみが機体に対して成分のみが機体に対してN/rev成分として伝達すること成分として伝達すること,
    またメインロータ回転面内方向のメインロータ回転面内方向の(N-1)/rev,ならびに(N+1)/rev荷重およびモーメントが
    N/rev荷重およびモーメントとして機体に伝達することに由来するものである.
    メインメインロータブレードルート部においてロータブレードルート部において1/rev以上の高調波成分以上の高調波成分の荷重
    およびモーメントの荷重およびモーメントはは以以下の理由により発生する下の理由により発生する.
    回転翼機が回転翼機が前進飛行する際に前進飛行する際に,メインロータブレード回転面へのインフロー分布が,
    回転するブレードブレード翼素の翼素の$\alpha$に影響を与えに影響を与える.
    結果として発生するルート部の荷重およびモーメントはるルート部の荷重およびモーメントは,
    メインロータブレードが弾性体であるがゆえに,N/rev成分成分((N=1,2,……))近傍に存在する固有振動モード近傍に存在する
    固有振動モードが励起するが励起することにより1/rev以上以上の高調波成分を含むこととなる.
    通常,初度の開発設計の中でその影響を小さくするよう設計されているものの発生自体を抑えることは困難である.
    特に面外方向の振動荷重は回転する各々1本に発生するN/rev荷重が胴体側にN/rev荷重として伝達する.
    このため,高調波成分も主として1/rev成分として見える荷重に含まれる高調波成分によるものとなる.
    N枚のメインロータブレードが厳密に管理された工程により形状,重量剛性のいずれも極めて均一な品質を保有していると仮定するならば,
    それ以外の高調波成分は機体座標系に伝達する際に各々キャンセルされ,結果として極めて小さな値となり,問題とはならない.
    ここで N/rev振動 は メインロータハブ中央位置における $x_{\mathrm{hub}}$,$y_{\mathrm{hub}}$,$z_{\mathrm{hub}}$座標系で定義されるN/rev成分の
    $F_{x\mathrm{hub}}$,$F_{y\mathrm{hub}}$,$F_{z\mathrm{hub}}$,$M_{x\mathrm{hub}}$,$M_{y\mathrm{hub}}$,$M_{z\mathrm{hub}}$から生ずるものでありこれらのN/rev成分は,その荷重あるいはモーメントが
    インプレーンかアウトオブプレーンかによってメインロータ回転系における振動成分が異なる.(表 1-1~表 1-6)
    回転翼機の振動は他にもシングルロータにおけるテールロータからの振動やエンジンや駆動系統からの振動,さらには操縦系統からのトランジェントな入力,
    ガスト応答などがあるもののメインロータ由来のN/rev振動と比較して十分に小さいものであるため本論文では議論の対象としない.

    \clearpage 

    \par
    表(1-1)~表(1-6)に示すメインロータハブ中央位置における荷重およびモーメントは回転するブレードが空力荷重を受けて
    結果的に固定側である胴体側に伝わる荷重をまとめており,回転系でのブレードルート部における荷重の説明を(図1-3)に示す.

    \clearpage

    \par
    ここで(図1-3)に示すように,$N$本ブレードのヘリコプタにおいて,$m$番目(アジマス角 $\Psi_m$に位置する)のブレードの
    ルート部における荷重は一般に定常成分と1/rev成分およびその高調波成分より構成される.
    例えば,$S_z$については,周波数ごとにその成分を記載することにより,(表 1-1)の左列の総和となる
    (6/rev以上は省略する.また (表1-1)は $F_zhub$について示しており,メインロータ回転面内の荷重である$F_xhub$,$F_yhub$については
    (表1-2)及び(表1-3)に,また,$M_xhub$,$M_yhub$および,$M_xhub$については(表 1-4~表 1-6)に示す.
    (表1-1)~(表1-6)は 左列に周波数ごとにまとめられている.
    回転しているブレードルート荷重が回転をしていないMRH座標系にどのように伝達するかをまとめている.
    これらは全てブレード枚数 N=4の結果であり,胴体系伝達後の荷重およびモーメントは(表1-1),(表1-6)の結果から分かるようにアウトオブプレーンの
    荷重$F_zhub$および モーメント$M_zhub$は 定常成分および4/rev成分が周波数の変調を起こすことなく,回転系から胴体系に伝わる.
    また,4/rev以外の高調波成分は0となる一方で,インプレーンの荷重およびモーメントは(表 1-2~表 1-5)から分かるように,
    回転系の3/rev,5/revの成分が変調して,胴体系に4/revとして伝達する.
    また,1/rev成分が定常成分として伝達し,それ以外の高調波成分は0となる.(なお,伝達後のアジマス角$\psi$は$m=1$のアジマス角である)
    それぞれの荷重モーメント成分の周波数の変調については文献10に示されている通りであるが
    特に変調後にキャンセルされて0とならないケースについては補遺Aに示す




    \section{本研究の目的}
    \par
    前節で述べた課題を解決するための方法として,可変ピッチロータを搭載することが考えられる.
    可変ピッチロータは,推力変化が速いことに加えて,推力調節をブレードピッチ角の変更で行うことで
    プロペラ回転数を高く保つことが可能となるため,機体の下降時など推力が低い状態でも
    大きなヨーイングモーメントを発生することができ,飛行の安定性の向上につながる.
    可変ピッチ搭載型ドローンの懸念点は,翼端失速を起こす限界ピッチ角であるが,この問題もブレードの
    アスペクト比を小さくすることで失速が発生する限界ピッチ角を大きくし,解消することができる.
    したがって,マルチロータ型ドローンでは低アスペクト比ブレードを高速回転で用いることで
    飛行の安定性は大きく向上する.一方で,低アスペクト比のロータを高速で回転させると
    ブレードに生じる遠心力による,ねじり下げモーメントが大きくなる.これは,可変ピッチ機構に用いる
    サーボアクチュエータの大型化や,カウンターウェイトの搭載を必要とするなど,設計上の制約と
    なることが知られている.そこで,本研究では最適なブレード設計を行うための基礎研究として,ブレードのアスペクト比
    やソリディティを変化させた場合のねじり下げモーメントの影響と搭載必要と考えられる
    カウンターウェイトについて考察する.

    \begin{figure}[H]
      \centering
      \includegraphics[keepaspectratio, width=0.6\linewidth]{photo/module.png}
      \caption[Module for controlling collective pitch]{Module for controlling collective pitch}
      \label{fig:sarbo}
    \end{figure}

    \begin{figure}[H]
      \centering
      \includegraphics[keepaspectratio, width=0.6\linewidth]{photo/collective_drone.jpg}
      \caption[Variable pitch multirotor aircraft AXM-1000]{Variable pitch multirotor aircraft AXM-1000 }
      \label{fig:collective}
    \end{figure}
    

    \section{本論文の構成}
    \par
    本論文は5章から構成される.第1章では,研究背景として,マルチロータ機の運用上の課題と,制御の
    応答の速さの重要性について示した.第2章では,低アスペクト比のブレードについて解析対象の個体モデルを
    決定する.ねじり下げモーメントの発生要因と検討対象となるブレード形状の特性について述べる.最後に,本研究で検討した
    ブレードモデルについて述べる.第3章では,ホバリング時の性能指数を用いたトリムの決定と,その際に発生するねじり下げモーメント
    について示す.トリム時に発生する遠心力によるねじり下げモーメントを相殺するカウンターウェイトを搭載し,推力を増加させた際の
    ねじり下げモーメントの変化について述べる.最後に,ねじり下げモーメントによって発生する消費パワーを計算し,サーボモータに与える
    負荷を検討する.第4章では, 最適なブレード設計を行うために考慮すべき点について述べ,その効果について考察する.
    まず,CFD解析より,空気力によるねじり下げモーメントについて述べる.遠心力によるねじり下げモーメントの重要性について考察する.
    次に翼型がねじり下げモーメントに与える影響について考察する.最後に,設計時のねじり下げがねじり下げモーメントに与える影響について考察する.
    第5章では,前節までの結果をまとめ,結論を示す.