\chapter{序論}
\label{chap:intro}
\begin{flushright}
	\begin{minipage}{0.8\hsize}
		\quad 本章では,本論文の研究背景と研究目的について説明する.
		まず,シングルロータヘリコプタの現状と運用上の課題であるN/revの振動荷重について説明する.
        次に,低振動化への従来の取り組みを示す.
        そして,本論文の提案手法である重心移動によるシングルロータヘリコプタの低振動化について述べる.
        最後に本研究で使用する解析ツールについて説明する.
	\end{minipage}
\end{flushright}
    \section{研究背景}
    \par
    航空機は一般に揚力を利用して飛行するが,固定翼機とは異なり,回転翼機はホバリングや鉛直方向の移動が可能である.
    この特性から,災害救助活動やドクターヘリによる患者輸送など,多様な応用が可能となっている.
    固定翼機における揚力は主翼によって発生するのに対し,回転翼機では一定角速度で回転する複数の翼(ブレード)によって揚力が発生する.
    これらのブレードは,メインロータとして機体の運動を制御する主機構を構成しており,通常2枚以上で構成される.
    メインロータブレードのピッチ角は,スワッシュプレートと呼ばれる機構を介して定常的にも周期的にも変更可能である.
    この機構により,機体の上昇・降下・前進・横進といった各方向の運動制御が行われる.
    しかしながら,メインロータが回転する際には必然的に機体に対してトルクが作用する.
    このトルクを打ち消すために,回転翼機には様々な様式がある.
    代表的な構成として,単一のメインロータと機体尾部のテールロータによってトルクを制御するシングルロータヘリコプタが挙げられる.
    一例として,SH-60K哨戒ヘリコプタ\ref{SH-60K(シングルロータ)}が挙げられる.
    さらに,前後方向に配置された2つのロータが互いのトルクを打ち消すことで機体を安定化させるタンデムロータヘリコプタが知られている.
    この形式には,CH-47J\ref{CH-47J(タンデムロータ)}のような機体がある.
    タンデムロータ機は,ロータが機体前後方向に存在するため,胴体の重心移動に対してロバスト性を有する特徴を持つ.
    また,二重反転式ロータヘリコプタとして,X2\ref{X2(二重反転式ロータ)}等の上下二段で反対方向に回転するロータを備えた形式が存在し,テールロータを不要としている.
    さらに,二重反転ロータの機構を簡素化し整備性を向上させた交差反転式ロータヘリコプタとして,K-MAX\ref{K-MAX(交差反転式ロータ)}のような機体も存在する.
    (Fig.\ref{fig:helicopter_types})
    これらいずれの回転翼機においても,前進飛行時にはロータ面が飛行速度ベクトルに対してほぼ平行となるエッジワイズフライトとなる.
    この状態は固定翼機におけるプロペラ運動とは大きく異なり,回転翼まわりの空気の流れに非軸対称性を生じさせ,振動の発生要因となる.
    例えば,シングルロータの上面図を(図 1-2)に示す.
    メインロータブレードに作用する空力荷重は,ブレードの位置を表すアジマス角$\Psi$に依存して異なる.
    特に,前進側ブレード($\Psi$ = 90 deg)と後退側ブレード($\Psi$ = 270 deg)では,ブレードに作用する相対速度が大きく異なる.
    前進側/後退側ブレードのRotor Station $r$ 位置の微小翼素が受けるロータ回転面内流体速度は次の式で表される.
    (ロータ回転面は$V_{c}$に対して平行とする.)

    \begin{equation}
        V = V_{c} + r\Omega (\Psi = 90 deg)
    \end{equation}
    
    \begin{equation}
        V = -V_{c} + r\Omega (_psi = 90 deg)
    \end{equation}

    \begin{figure}[htbp]
    \centering

    % 1行目
    \begin{subfigure}[b]{0.45\linewidth}
        \centering
        \includegraphics[width=\linewidth]{photo/sh60k.jpg}
        \caption{SH-60K(シングルロータ)}
    \end{subfigure}
    \hfill
    \begin{subfigure}[b]{0.45\linewidth}
        \centering
        \includegraphics[width=\linewidth]{photo/CH_47J.jpg}
        \caption{CH-47J(タンデムロータ)}
    \end{subfigure}

    \vspace{5mm}

    % 2行目
    \begin{subfigure}[b]{0.45\linewidth}
        \centering
        \includegraphics[width=\linewidth]{photo/X2.jpg}
        \caption{X2(二重反転式ロータ)}
    \end{subfigure}
    \hfill
    \begin{subfigure}[b]{0.45\linewidth}
        \centering
        \includegraphics[width=\linewidth]{photo/KMAX.jpg}
        \caption{K-MAX(交差反転式ロータ)}
    \end{subfigure}

    \caption{ヘリコプタの様式\cite{SH60K,CH47J,X2,KMAX}}
    \label{fig:helicopter_types}
    \end{figure}


    \begin{figure}[H]
      \centering
      \includegraphics[keepaspectratio, width=0.8\linewidth]{photo/forwardflight.png}
      \caption{前進飛行時のシングルロータ (上面図 )\cite{Prouty2001}}
      \label{fig:forwardflight}
    \end{figure}


    \par
    高速で前進飛行する場合において,N/rev振動(N:ブレード枚数)が卓越する結果となる.
    このN/rev振動は メインロータブレードに働くメインロータ回転面外方向の荷重およびモーメントのうち
    N/rev成分のみが機体に対して成分のみが機体に対してN/rev成分として伝達すること成分として伝達すること,
    またメインロータ回転面内方向のメインロータ回転面内方向の(N-1)/rev,ならびに(N+1)/rev荷重およびモーメントが
    N/rev荷重およびモーメントとして機体に伝達することに由来するものである.
    メインメインロータブレードルート部においてロータブレードルート部において1/rev以上の高調波成分以上の高調波成分の荷重
    およびモーメントの荷重およびモーメントは以下の理由により発生する.
    回転翼機が回転翼機が前進飛行する際に前進飛行する際に,メインロータブレード回転面へのインフロー分布が,
    回転するブレードブレード翼素の翼素の$\alpha$に影響を与えにる.
    結果として発生するルート部の荷重およびモーメントはるルート部の荷重およびモーメントは,
    メインロータブレードが弾性体であるがゆえに,N/rev成分成分((N=1,2,……))近傍に存在する固有振動モード近傍に存在する
    固有振動モードが励起するが励起することにより1/rev以上以上の高調波成分を含むこととなる.
    通常,初度の開発設計の中でその影響を小さくするよう設計されているものの発生自体を抑えることは困難である.
    特に面外方向の振動荷重は回転する各々1本に発生するN/rev荷重が胴体側にN/rev荷重として伝達する.
    このため,高調波成分も主として1/rev成分として見える荷重に含まれる高調波成分によるものとなる.
    N枚のメインロータブレードが厳密に管理された工程により形状,重量剛性のいずれも極めて均一な品質を保有していると仮定するならば,
    それ以外の高調波成分は機体座標系に伝達する際に各々キャンセルされ,結果として極めて小さな値となり,問題とはならない.
    ここで N/rev振動 は メインロータハブ中央位置における $x_{\mathrm{hub}}$,$y_{\mathrm{hub}}$,$z_{\mathrm{hub}}$座標系で定義されるN/rev成分の
    $F_{x\mathrm{hub}}$,$F_{y\mathrm{hub}}$,$F_{z\mathrm{hub}}$,$M_{x\mathrm{hub}}$,$M_{y\mathrm{hub}}$,$M_{z\mathrm{hub}}$から生ずるものであり
    これらのN/rev成分は,その荷重あるいはモーメントがインプレーンかアウトオブプレーンかによってメインロータ回転系における振動成分が異なる.
    (Table.\ref{table:Sz_to_Fzhub}~Table.\ref{table:NL_to_Mzhub})
    回転翼機の振動は他にもシングルロータにおけるテールロータからの振動やエンジンや駆動系統からの振動,さらには操縦系統からのトランジェントな入力,
    ガスト応答などがあるもののメインロータ由来のN/rev振動と比較して十分に小さいものであるため本論文では議論の対象としない.

    \clearpage 

    \par
    Table.\ref{table:Sz_to_Fzhub}~Table.\ref{table:NL_to_Mzhub}に示すメインロータハブ中央位置における荷重およびモーメントは回転するブレードが空力荷重を受けて
    結果的に固定側である胴体側に伝わる荷重をまとめており,回転系でのブレードルート部における荷重の説明をFig.\ref{fig:MRHFM}に示す.
    
    \begin{figure}[H]
      \centering
      \includegraphics[keepaspectratio, width=0.5\linewidth]{photo/MRHFM.png}
      \caption{回転系におけるメインロータブレードルート部荷重 およびモーメント}
      \label{fig:MRHFM}
    \end{figure}
    
    
    ここでFig.\ref{fig:MRHFM}に示すように,$N$本ブレードのヘリコプタにおいて,$m$番目(アジマス角 $\Psi_m$に位置する)のブレードの
    ルート部における荷重は一般に定常成分と1/rev成分およびその高調波成分より構成される.
    例えば,$S_z$については,周波数ごとにその成分を記載することにより,Table.\ref{table:Sz_to_Fzhub}の左列の総和となる
    (6/rev以上は省略する.)またTable.\ref{table:Sz_to_Fzhub}は$F_{z\mathrm{hub}}$について示しており,メインロータ回転面内の荷重である$F_{x\mathrm{hub}}$,$F_{y\mathrm{hub}}$については
    Table.\ref{table:SxSr_to_Fxhub}及びTable.\ref{table:SxSr_to_Fyhub}に,また,$M_{x\mathrm{hub}}$,$M_{y\mathrm{hub}}$および,$M_{z\mathrm{hub}}$については
    (Table.\ref{table:NF_to_Mxhub}~Table.\ref{table:NL_to_Mzhub})に示す.
    Table.\ref{table:Sz_to_Fzhub}~Table.\ref{table:NL_to_Mzhub}は 左列に周波数ごとにまとめられている.
    回転しているブレードルート荷重が回転をしていないMRH座標系にどのように伝達するかをまとめている.
    これらは全てブレード枚数 N=4の結果であり,胴体系伝達後の荷重およびモーメントはTable.\ref{table:Sz_to_Fzhub},Table.\ref{table:NL_to_Mzhub}の結果から分かるように
    アウトオブプレーンの荷重$F_{z\mathrm{hub}}$および モーメント$M_{z\mathrm{hub}}$は 定常成分および4/rev成分が周波数の変調を起こすことなく,回転系から胴体系に伝わる.
    また,4/rev以外の高調波成分は0となる一方で,インプレーンの荷重およびモーメントは(Table.\ref{table:SxSr_to_Fyhub}~Table.\ref{table:NF_to_Myhub})から分かるように,
    回転系の3/rev,5/revの成分が変調して,胴体系に4/revとして伝達する.
    また,1/rev成分が定常成分として伝達し,それ以外の高調波成分は0となる.(なお,伝達後のアジマス角$\psi$は$m=1$のアジマス角である)
    それぞれの荷重モーメント成分の周波数の変調については(文献10)に示されている通りであるが
    特に変調後にキャンセルされて0とならないケースについては(補遺A)に示す

    \begin{table}[H]
        \centering
        \caption{$S_z$各周波数成分の胴体系伝達後の$F_{z\mathrm{hub}}$}
        \label{table:Sz_to_Fzhub}
        \setlength{\tabcolsep}{5pt}
        \begin{tabular}{|c|c|c|} \hline
            周波数 &  $S_z$荷重成分 &   MRH荷重$F_{z\mathrm{hub}}$ N=4 \\ \hline
            定常 &  $S_{z0}$ & $4S_{z0}$ \\ \hline
            \raisebox{-1.2ex}{1/rev} & $S_{z1c}\cos\Psi_m$ & $0$ \\   \cline{2-3}
                                   & $S_{z1s}\sin\Psi_m$ & $0$ \\ \hline
            \raisebox{-1.2ex}{2/rev} & $S_{z2c}\cos 2\Psi_m$  & $0$ \\   \cline{2-3}
                                   & $S_{z2s}\sin 2\Psi_m$  & $0$ \\ \hline
            \raisebox{-1.2ex}{3/rev} & $S_{z3c}\cos 3\Psi_m$  & $0$ \\   \cline{2-3}
                                   & $S_{z3s}\sin 3\Psi_m$  & $0$ \\ \hline
            \raisebox{-1.2ex}{4/rev} & $S_{z4c}\cos 4\Psi_m$  & $4S_{z4c}\cos 4\psi$ \\   \cline{2-3}
                                   & $S_{z4s}\sin 4\Psi_m$  & $4S_{z4s}\sin 4\Psi$ \\ \hline
            \raisebox{-1.2ex}{5/rev} & $S_{z5c}\cos 5\Psi_m$  & $0$ \\    \cline{2-3}
                                   & $S_{z5s}\sin 5\Psi_m$  & $0$ \\ \hline
            \raisebox{-1.2ex}{6/rev} & $S_{z6c}\cos 6\Psi_m$  & $0$ \\    \cline{2-3}
                                   & $S_{z6s}\sin 6\Psi_m$  & $0$ \\ \hline
        \end{tabular}
    \end{table}


    \begin{table}[H]
        \centering
        \caption{$S_x$,$S_r$各周波数成分の胴体系伝達後の$F_{x\mathrm{hub}}$}
        \label{table:SxSr_to_Fxhub}
        \setlength{\tabcolsep}{5pt}
        \begin{tabular}{|c|c|c|c|} \hline
            周波数 &  $S_x$荷重成分 & $S_r$荷重成分 &  MRH荷重$F_{x\mathrm{hub}}$ N=4 \\ \hline
            定常 &  $S_{x0}$ &  $S_{r0}$ & $0$ \\ \hline
            \raisebox{-1.2ex}{1/rev} & $S_{x1c}\cos\Psi_m$ & $S_{r1c}\cos\Psi_m$ & $2S_{r1c}$ \\   \cline{2-4}
                                   & $S_{x1s}\sin\Psi_m$ & $S_{r1s}\sin\Psi_m$ & $2S_{r1s}$ \\ \hline
            \raisebox{-1.2ex}{2/rev} & $S_{x2c}\cos 2\Psi_m$  & $S_{r2c}\cos 2\Psi_m$  & $0$ \\   \cline{2-4}
                                   & $S_{x2s}\sin 2\Psi_m$  & $S_{r2s}\sin 2\Psi_m$  & $0$ \\ \hline
            \raisebox{-1.2ex}{3/rev} & $S_{x3c}\cos 3\Psi_m$  & $S_{r3c}\cos 3\Psi_m$  & $2S_{x3c}\sin 4\Psi + 2S_{r3c}\cos 4\Psi$ \\   \cline{2-4}
                                   & $S_{x3s}\sin 3\Psi_m$  & $S_{r3s}\sin 3\Psi_m$  & $-2S_{x3s}\cos 4\Psi + 2S_{r3s}\sin 4\Psi$ \\ \hline
            \raisebox{-1.2ex}{4/rev} & $S_{x4c}\cos 4\Psi_m$  & $S_{r4c}\cos 4\Psi_m$  & $0$ \\   \cline{2-4}
                                   & $S_{x4s}\sin 4\Psi_m$  & $S_{r4s}\sin 4\Psi_m$  & $0$ \\ \hline
            \raisebox{-1.2ex}{5/rev} & $S_{x5c}\cos 5\Psi_m$  & $S_{r5c}\cos 5\Psi_m$  & $-2S_{x5c}\sin 4\Psi + 2S_{r5c}\cos 4\Psi$ \\    \cline{2-4}
                                   & $S_{x5s}\sin 5\Psi_m$  & $S_{r5s}\sin 5\Psi_m$  & $2S_{x5s}\cos 4\Psi + 2S_{r5s}\sin 4\Psi$ \\ \hline
            \raisebox{-1.2ex}{6/rev} & $S_{x6c}\cos 6\Psi_m$  & $S_{r6c}\cos 6\Psi_m$  & $0$ \\    \cline{2-4}
                                   & $S_{x6s}\sin 6\Psi_m$  & $S_{r6s}\sin 6\Psi_m$  & $0$ \\ \hline
        \end{tabular}
    \end{table}

    \begin{table}[H]
        \centering
        \caption{$S_x$,$S_r$各周波数成分の胴体系伝達後の$F_{y\mathrm{hub}}$}
        \label{table:SxSr_to_Fyhub}
        \setlength{\tabcolsep}{5pt}
        \begin{tabular}{|c|c|c|c|} \hline
            周波数 &  $S_x$荷重成分 & $S_r$荷重成分 &  MRH荷重$F_{y\mathrm{hub}}$ N=4 \\ \hline
            定常 &  $S_{x0}$ &  $S_{r0}$ & $0$ \\ \hline
            \raisebox{-1.2ex}{1/rev} & $S_{x1c}\cos\Psi_m$ & $S_{r1c}\cos\Psi_m$ & $-2S_{r1c}$ \\   \cline{2-4}
                                   & $S_{x1s}\sin\Psi_m$ & $S_{r1s}\sin\Psi_m$ & $-2S_{r1s}$ \\ \hline
            \raisebox{-1.2ex}{2/rev} & $S_{x2c}\cos 2\Psi_m$  & $S_{r2c}\cos 2\Psi_m$  & $0$ \\   \cline{2-4}
                                   & $S_{x2s}\sin 2\Psi_m$  & $S_{r2s}\sin 2\Psi_m$  & $0$ \\ \hline
            \raisebox{-1.2ex}{3/rev} & $S_{x3c}\cos 3\Psi_m$  & $S_{r3c}\cos 3\Psi_m$  & $-2S_{x3c}\cos 4\Psi + 2S_{r3c}\sin 4\Psi$ \\   \cline{2-4}
                                   & $S_{x3s}\sin 3\Psi_m$  & $S_{r3s}\sin 3\Psi_m$  & $-2S_{x3s}\sin 4\Psi - 2S_{r3s}\cos 4\Psi$ \\ \hline
            \raisebox{-1.2ex}{4/rev} & $S_{x4c}\cos 4\Psi_m$  & $S_{r4c}\cos 4\Psi_m$  & $0$ \\   \cline{2-4}
                                   & $S_{x4s}\sin 4\Psi_m$  & $S_{r4s}\sin 4\Psi_m$  & $0$ \\ \hline
            \raisebox{-1.2ex}{5/rev} & $S_{x5c}\cos 5\Psi_m$  & $S_{r5c}\cos 5\Psi_m$  & $-2S_{x5c}\cos 4\Psi - 2S_{r5c}\sin 4\Psi$ \\    \cline{2-4}
                                   & $S_{x5s}\sin 5\Psi_m$  & $S_{r5s}\sin 5\Psi_m$  & $-2S_{x5s}\sin 4\Psi + 2S_{r5s}\cos 4\Psi$ \\ \hline
            \raisebox{-1.2ex}{6/rev} & $S_{x6c}\cos 6\Psi_m$  & $S_{r6c}\cos 6\Psi_m$  & $0$ \\    \cline{2-4}
                                   & $S_{x6s}\sin 6\Psi_m$  & $S_{r6s}\sin 6\Psi_m$  & $0$ \\ \hline
        \end{tabular}
    \end{table}

    \begin{table}[H]
        \centering
        \caption{$N_F$各周波数成分の胴体系伝達後の$M_{x\mathrm{hub}}$}
        \label{table:NF_to_Mxhub}
        \setlength{\tabcolsep}{5pt}
        \begin{tabular}{|c|c|c|} \hline
            周波数 &  $N_F$荷重成分 &   MRH荷重$M_{x\mathrm{hub}}$ N=4 \\ \hline
            定常 &  $N_{F0}$ & $0$ \\ \hline
            \raisebox{-1.2ex}{1/rev} & $N_{F1c}\cos\Psi_m$ & $0$ \\   \cline{2-3}
                                   & $N_{F1s}\sin\Psi_m$ & $2N_{F1s}$ \\ \hline
            \raisebox{-1.2ex}{2/rev} & $N_{F2c}\cos 2\Psi_m$  & $0$ \\   \cline{2-3}
                                   & $N_{F2s}\sin 2\Psi_m$  & $0$ \\ \hline
            \raisebox{-1.2ex}{3/rev} & $N_{F3c}\cos 3\Psi_m$  & $2N_{F3c}\sin 4\psi$ \\   \cline{2-3}
                                   & $N_{F3s}\sin 3\Psi_m$  & $-2N_{F3s}\cos 4\psi$ \\ \hline
            \raisebox{-1.2ex}{4/rev} & $N_{F4c}\cos 4\Psi_m$  & $0$ \\   \cline{2-3}
                                   & $N_{F4s}\sin 4\Psi_m$  & $0$ \\ \hline
            \raisebox{-1.2ex}{5/rev} & $N_{F5c}\cos 5\Psi_m$  & $-2N_{F5c}\sin 4\psi$ \\    \cline{2-3}
                                   & $N_{F5s}\sin 5\Psi_m$  & $2N_{F5s}\cos 4\psi$ \\ \hline
            \raisebox{-1.2ex}{6/rev} & $N_{F6c}\cos 6\Psi_m$  & $0$ \\    \cline{2-3}
                                   & $N_{F6s}\sin 6\Psi_m$  & $0$ \\ \hline
        \end{tabular}
    \end{table}

    \begin{table}[H]
        \centering
        \caption{$N_F$各周波数成分の胴体系伝達後の$M_{y\mathrm{hub}}$}
        \label{table:NF_to_Myhub}
        \setlength{\tabcolsep}{5pt}
        \begin{tabular}{|c|c|c|} \hline
            周波数 &  $N_F$荷重成分 &   MRH荷重$M_{y\mathrm{hub}}$ N=4 \\ \hline
            定常 &  $N_{F0}$ & $0$ \\ \hline
            \raisebox{-1.2ex}{1/rev} & $N_{F1c}\cos\Psi_m$ & $-2N_{F1c}$ \\   \cline{2-3}
                                   & $N_{F1s}\sin\Psi_m$ & $0$ \\ \hline
            \raisebox{-1.2ex}{2/rev} & $N_{F2c}\cos 2\Psi_m$  & $0$ \\   \cline{2-3}
                                   & $N_{F2s}\sin 2\Psi_m$  & $0$ \\ \hline
            \raisebox{-1.2ex}{3/rev} & $N_{F3c}\cos 3\Psi_m$  & $-2N_{F3c}\cos 4\psi$ \\   \cline{2-3}
                                   & $N_{F3s}\sin 3\Psi_m$  & $-2N_{F3s}\sin 4\psi$ \\ \hline
            \raisebox{-1.2ex}{4/rev} & $N_{F4c}\cos 4\Psi_m$  & $0$ \\   \cline{2-3}
                                   & $N_{F4s}\sin 4\Psi_m$  & $0$ \\ \hline
            \raisebox{-1.2ex}{5/rev} & $N_{F5c}\cos 5\Psi_m$  & $-2N_{F5c}\cos 4\psi$ \\    \cline{2-3}
                                   & $N_{F5s}\sin 5\Psi_m$  & $-2N_{F5s}\sin 4\psi$ \\ \hline
            \raisebox{-1.2ex}{6/rev} & $N_{F6c}\cos 6\Psi_m$  & $0$ \\    \cline{2-3}
                                   & $N_{F6s}\sin 6\Psi_m$  & $0$ \\ \hline
        \end{tabular}
    \end{table}

    \begin{table}[H]
        \centering
        \caption{$N_L$各周波数成分の胴体系伝達後の$M_{z\mathrm{hub}}$}
        \label{table:NL_to_Mzhub}
        \setlength{\tabcolsep}{5pt}
        \begin{tabular}{|c|c|c|} \hline
            周波数 &  $N_L$荷重成分 &   MRH荷重$M_{z\mathrm{hub}}$ N=4 \\ \hline
            定常 &  $N_{L0}$ & $4N_{L0}$ \\ \hline
            \raisebox{-1.2ex}{1/rev} & $N_{L1c}\cos\Psi_m$ & $0$ \\   \cline{2-3}
                                   & $N_{L1s}\sin\Psi_m$ & $0$ \\ \hline
            \raisebox{-1.2ex}{2/rev} & $N_{L2c}\cos 2\Psi_m$  & $0$ \\   \cline{2-3}
                                   & $N_{L2s}\sin 2\Psi_m$  & $0$ \\ \hline
            \raisebox{-1.2ex}{3/rev} & $N_{L3c}\cos 3\Psi_m$  & $0$ \\   \cline{2-3}
                                   & $N_{L3s}\sin 3\Psi_m$  & $0$ \\ \hline
            \raisebox{-1.2ex}{4/rev} & $N_{L4c}\cos 4\Psi_m$  & $4N_{L4c}\cos 4\psi$ \\   \cline{2-3}
                                   & $N_{L4s}\sin 4\Psi_m$  & $4N_{L4s}\sin 4\Psi$ \\ \hline
            \raisebox{-1.2ex}{5/rev} & $N_{L5c}\cos 5\Psi_m$  & $0$ \\    \cline{2-3}
                                   & $N_{L5s}\sin 5\Psi_m$  & $0$ \\ \hline
            \raisebox{-1.2ex}{6/rev} & $N_{L6c}\cos 6\Psi_m$  & $0$ \\    \cline{2-3}
                                   & $N_{L6s}\sin 6\Psi_m$  & $0$ \\ \hline
        \end{tabular}
    \end{table}


    \clearpage

    \section{低振動化に対する取り組み}
    \par
    回転翼機が高速で前進巡航飛行しているとき,メインロータハブに働く振動荷重及びモーメントが大きくなることは前項で述べたとおりである.
    このことは プライマリサーボよりメインロータ側に存在する全てのメインロータダイナミックコンポーネントの構成品の疲労寿命の減少,
    更に振動の下流である胴体構造の疲労強度の低下に繋がる.
    また,その振動荷重及びモーメントが胴体を加振することにより生ずるパイロット座席での加速度がパイロット人体の疲労や不快感を生起することとなる(11-16)
    さらには,エンジン,装備品,計器類,航法支援装置や武装システムといった機器の機能上の環境振動における許容値を超えてしまうおそれも生じる.
    低振動化に取り組まなければ,上述のダイナミックコンポーネントや胴体構造の疲労強度の低下に対しては部品の肉厚の増加などで対応せざるを得ないため,
    重量増加を招くこと人体の疲労に関してはミッションのパフォーマンス低下につながること,装備品の誤作動や機能停止に関しては
    飛行安全そのものを脅かすこととなってしまう.
    近年,MILスタンダードの振動要求値は改訂版が発行される度に段階的に下がっており,より低振動に対する要求が高まっていることや,
    更には,今後コンパウンドヘリコプタに代表される高速で飛行できる回転翼機の需要が高まっていくことが予想されるため(17),
    高速巡航時の振動低減技術はより一層重要なものとなっていくことが予想される.

    \par
    胴体の振動を低下させる方法には大きく2種類の様式がある.
    それは,胴体に代表される固定系での対策とメインロータブレードを代表とする回転系における対策の2つである.
    前者については代表的なものとしてダイナミックバイブレーションアブソーバー(DVA やアクティブバイブレーションコントロール(AVCが挙げられる (18)図1-4
    一般に,DVAが十分な制振能力を発揮するためには機体質量の約2倍の動マスを保有したDVAを搭載する必要があるといわれており,航空機に搭載する装備品としては好ましくない.
    また 固定系での対策としては他にいわゆる"nodamatic system"(19)といわれる振動絶縁装置がある.(図1-5)に示すように MGBの胴体取付部に梁の曲げで荷重伝達をする機構を設けて
    MGB取付部においてモード形の節の位置になるよう,あらかじめ設計しておく方法である.
    図 1-6には Dynamic Antiresonant Vibration Isolator(DAVI)(20)-(22)の 装置写真 を示す.
    DAVIは ”nodamatic system”と同様に,MGBと胴体をつなぐ荷重伝達部においてバネとウェイトから構成されており 固有振動数を応答が小さくなる反共振点になるよう設計し,
    振動の伝達を小さくするよう考えられたものである.類似の例として The Improved Rotor Isolation System (IRIS)(23)The hydraulic antiresonant isolator(24)
    さらにはThe Liquid Inertia Vibration Elimination (LIVE)(25)などが挙げられるが,ここでは詳細は割愛する.

    \par
    一方,後者に関してはパッシブな手法としてメインロータブレードのルート部にペンデュラム(遠心振り子)やメインロータハブ 中央位置にバイファイラを装着する手法が一般的である.(図 1-7 図 1-8)
    ペンデュラムは (図 1-1)で示した $F_z$成分を低減するため,N/revに またバイファイラは同じく図 1-1で示した $F_{x\mathrm{hub}}$,$F_{y\mathrm{hub}}$成分を低減するため(N-1)/rev,
    あるいは (N+1)/revに合わせて共振点を調整することが基本的である.

    \par
    このうちバイファイラに関しては遠心振り子のようなパッシブな装置ではなく,アクティブな装置についても試作品が完成しており,パッシブなバイファイラよりも重量軽減することが実現している.
    また,飛行試験でも効果が確認されているが実用には至っていない.(HMVS: Hub Mounting Vibration Suppression)
    ここまで述べた振動低減手法についてはいずれも振動低減の効果を十分得るためにはそれなりの質量が必要であり航空機に搭載するうえで不利益となる.
    さらに,調整が非常にセンシティブであり,メインロータ回転数の僅かな変動でもその振動低減効果に有意差が生じるため,その調整に非常に多くの労力を払う必要がある.
    また,振動低減のために増加した質量を削減するためにはアクティブ化等の努力が必要となり,結果としてコスト増を招くため,運用者側に対する負担が大きいものとなる.


    \par
    回転系における対策には代表的なものがもう1種類あり,それはTEF(Trailing Edge Flaps)と呼ばれる手法であり,図1-9にその概念図を示す.
    これは初期のアイディアはStraub and Charles(28)により紹介され Ormiston(29)がそれに続いた.
    TEFは事前の数値解析やその実証試験において非常に大きな努力が必要であり,当初の目標であったスワッシュプレートレス設計を実現させるためにはTEFを
    駆動するスマート材料では十分な変位量を確保することが難しい(30ー34)こと,また回転系のTEFに胴体側固定系からスリップリングを介して十分な電力を供給することにも技術的な困難を伴うことが生じた.
    しかしながら,TEFはその技術的課題を克服することを待たなくてもプライマリコントロールのみならず振動低減のためのデバイスとしてとても有望なものであった.
    4枚ブレードの回転翼機においてメインロータハブ中央位置での$x_{\mathrm{hub}}$,$y_{\mathrm{hub}}$,$z_{\mathrm{hub}}$座標系における$F_{z\mathrm{hub}}$や
    $M_{x\mathrm{hub}}$,$M_{y\mathrm{hub}}$の 4/rev振動荷重及びモーメントが90\%低減することが解析的に求められた.(35-36)
    しかし実際には上述の通り,TEFを駆動するアクチュエータの変位量が不足しておりここまでの振動低減は今のところ望めない.
    そのうえ,空力弾性学上の問題に関して慎重な検討が必要であることが分かった.(37)

    \par
    ここまで述べたように回転翼機の振動を低減させるためにはそれなりの質量増加を覚悟する必要があること,またその調整に多くの時間を費やし,革新的な技術の進展を待たねばならない項目もある.
    生来振動の少ない機体とするにはどうすればよいのだろうか.

    \par
    先行する研究としてGandhi, F. S. and Sekura, M. K. (38)は 回転翼機の後方に位置する水平尾翼の舵角を変更してメインロータハブ中央位置に
    定常的なモーメント($M_x$,$M_y$)及び$F_z$を 付与した際の振動の変化を解析的に検討した.
    UH-60及びBo-105の2種類の機体において,UMARCを用いた解析を実施してメインロータ中央位置に働く4/rev振動荷重及びモーメントを検討した.
    6つの成分$F_{x\mathrm{hub}}$,$F_{y\mathrm{hub}}$,$F_{z\mathrm{hub}}$及び$M_{x\mathrm{hub}}$,$M_{y\mathrm{hub}}$,$M_{z\mathrm{hub}}$の無次元化値の
    2乗平均値 J(Vibration Index)が評価指標とされ,その値で結果が整理された.
    その結果,ベースラインのスタビレータ舵角セッティングとは異なる角度で振動が極小値をとることが判明した.
    論文 (38)では高速巡航時での検討にとどまるが,中速域においても振動は小さくないことが分かっており,さらに中速領域ではスタビレータに作用する空力荷重が小さくなるため,
    スタビレータ舵角の変更による低振動化には十分な効果が得られない可能性が考えられた.

    

    \section{本論文の目的}
    \par
    本論文では重心移動による振動低減の可能性について検討を実施した.
    すでに,小型~中型機においてはサイクリックピッチの代替手法として重心移動により機体のコントロールが可能であることが解析的に検討されている (39,40)
    本論文では,まずは第2章においてメインロータハブ 中央位置に作用する振動荷重及びモーメントを低減させるために機体の重心位置を前後左右に移動させることで
    メインロータハブ中央位置に機体座標系における定常的な$M_x$,$M_y$を作用させることを検討した.
    重心移動は機体内部の装備品を移動させることで実現させることを考えており,可動式としては考えない.
    重心移動により付加することができる$M_x$,$M_y$は機体速度によらず高速域以外でも振動低減できることが期待できる.
    本解析のために使用するツールはCAMRAD $\mathrm{II}$とし,このツールでトリム解析を実施した結果について考察を加え,解析結果の妥当性について論じる.
    CAMRAD $\mathrm{II}$ではこのトリム解析においては様々な重心位置において機体重心位置における荷重およびモーメントのバランスを解いているが,
    この時のメインロータハブ中央位置に作用する空力による振動荷重およびモーメントも導出される.
    第3章では第2章で得られた振動荷重及びモーメントを示す.
    また,パイロット座席位置を機体振動の代表的な評価点として考え,その位置での加速度を解析的に示す.
    CAMRAD $\mathrm{II}$にて得られたメインロータハブ中央位置での振動荷重およびモーメントを使用してパイロット座席位置における加速度応答を数値解析で求め,考察を加える.
    また,振動低減が起きる理由についても考察する.
    加えて,機体の重心移動に関しては可動方式を採らないことから,重心移動により変化するトリムが与える影響を検討する必要がある.
    第4章では重心移動がメインロータの必要パワーに与える影響と機体の固有安定性に与える影響を検討する.
    第5章では振動低減のために必要な重心移動が機体内部の装備品の移動だけでは対応できない状況も想定して重心移動の代替手法として補助翼搭載(前後 左右方向)の効果についても検討を実施する.


    \section{使用する解析ツール}
    \par
    本論文では回転翼機の振動と機体の重心位置の関係性について解析的に検討を実施した.
    その際に機体のトリム姿勢角やメインロータハブ中央位置に働く振動荷重及びモーメントならびにメインロータブレードが1回転する間の各アジマス角での様々な値が得られる.
    これまで包括的なヘリコプタの解析ツールは産官学において様々開発されている.
    図 1-10にその解析ツールと開発者と開発時期について示す.
    本論文で使用する解析ツールはこれまで回転翼機の振動解析に数多く使用され実績のあるCAMRAD $\mathrm{II}$(41)を使用する.
    CAMRAD $\mathrm{II}$はシングルロータヘリコプタにとどまらず,さまざまな形態の回転翼機の全体の空力解析ツールである.
    もともとは NASAに所属するWayne Johnsonが NASA,そして米陸軍のためにCAMRADを開発した.
    CAMRADは1978年から1979年にかけて開発された.
    それまでは回転翼機を包括的に解析できる技術は存在しておらず,個別の用途に合わせて単独の解析を実施していた.
    メインロータ,テールロータから成るシングルロータヘリコプタをブレードについて,梁要素として取り扱い,
    Scully Vortex Wake Modelによるインフロー計算を用いていた.
    このあと,1986年から1989年にかけて,CAMRADはCAMRAD/JAとしてJohnson Aeronauticsがさらに改良を加えることとなり,特にインフロー計算における精度向上を果たした.
    ここで開発された自由渦法( Free Wake Model)は COPTER, UMARC, 2GCHASといった同種のソフトウェアにも組み込まれている.
    また,回転翼機の解析は複雑で複数の力学の分野を考慮に入れておかなければならない.
    この後CAMRADはCAMRAD $\mathrm{II}$として最新版へと変化を遂げることになったが,このツールは回転翼機のメインロータブレードを含む胴体やドライブシャフトを弾性体として扱う
    マルチボディダイナミクス,そして回転翼の空気力学を取り込んだものである.
    解析自体は3つの大きなタスク(トリム解析,非定常解析,フラッター解析)で構成されている.
    このように CAMRAD $\mathrm{II}$はメインロータハブ位置における振動荷重やモーメントを精度よく解析することが可能 (42)となっており,本論文でも評価する過渡応答解析や安定性解析を実施することができる.
    本論文では,トリム解析においてメインロータ回転面を通過するインフローを求める際 に,自由後流モデル( Free Wake Model)を使用した.




    \begin{figure}[H]
      \centering
      \includegraphics[keepaspectratio, width=0.8\linewidth]{photo/CAMRADhistory.png}
      \caption{回転翼機の包括的解析ツール\cite{Johnson1998}}
      \label{fig:}
    \end{figure}
