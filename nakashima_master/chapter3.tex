\chapter{CAMRAD $\mathrm{II}$による解析}
\label{chap:camrad_analysis}
\begin{flushright}
	\begin{minipage}{0.8\hsize}
		\quad 本章では,飛行試験で用いたラジコンヘリコプタについてCAMRAD $\mathrm{II}$による解析結果を示す.
		      まず,飛行試験とによる結果の違いについて述べる.
			  次に,ブレード方位角$\Psi$と有効迎角$\alpha$の関係や,有効迎角 $\alpha$と揚力係数$C_l$の関係などのグラフから,低振動化に寄与する要素を特定する.
              最後に,$r/R= 0.855$における$\alpha U^2$,$C_l U^2$の振幅の値をFFTにより求め,支配的な周波数成分の組み合わせについて述べる.
        
	\end{minipage}
\end{flushright}

    \section{CAMRAD $\mathrm{II}$と飛行試験の$F_{z\mathrm{hub}}$の比較}
    \label{sec:camrad_result}
    \par
    飛行試験の結果に加えて,ラジコンヘリコプタのCAMRAD $\mathrm{II}$による解析を実施した.
    Fig.\ref{fig:camrad_flight_fzhub}に解析結果を示す.
    黒色の棒グラフが4/rev振動に強い相関のあるハブでの$z$方向の外力$F_{z\mathrm{hub}}$の4/rev成分であり,
    灰色の棒グラフが飛行試験から得られた$a_z$と機体質量$m$の積を示している.
    飛行試験と同様にcase $b$,case $c$はcase $a$よりも$F_{z\mathrm{hub}}$の4/rev成分が低減していることが確認できる.
    振動が増加すると予想していたcase $d$もcase $a$より$F_{z\mathrm{hub}}$の4/rev成分が大きい.

    \begin{figure}[H]
      \centering
      \includegraphics[keepaspectratio, width=0.5\linewidth]{photo/camrad_flight_fzhub.png}
      \caption{CAMRAD $\mathrm{II}$と飛行試験の$F_{z\mathrm{hub}}$の比較}
      \label{fig:camrad_flight_fzhub}
    \end{figure}

    \section{有効迎角$\alpha$と揚力係数$C_l$,誘導速度$v_i$について}
    \label{sec:psi_alpha_cl_vi}

    \par
    Fig.\ref{fig:psi_alpha}には各ケースにおけるブレード方位角$\Psi$と有効迎角$\alpha$の関係を示す.
    case $a$と比較した際,重心を左舷後方にした場合のcase $b$とcase $c$では$\alpha$の振幅が小さくなっている.
    一方,重心を右舷後方においたcase $d$においては$\alpha$の振幅が大きくなっている.
    また,どのケースにおいても$\Psi=300°$付近で$\alpha$の減少が見られる.
    Fig.\ref{fig:alpha_cl}には,有効迎角 $\alpha$と揚力係数$C_l$の関係を示す.
    $C_l$は$\alpha$と翼素の揚力の非定常効果によって決まる.
    有効迎角は準定常空気力に対応することに留意されたい. 
    どのケースにおいても$r/R= 0.750$より$r/R= 0.855$で非線形性が大きい.
    しかしながら、$C_l$における非定常効果によって生じるこの非線形性は大きくない.
    なぜなら$\alpha$は失速角よりも十分に小さいためである.
    有効迎角$\alpha$の最大値はcase $b$とcase $c$で約4°であり,case $a$では約5°,case $d$では約5.5°である.
    これらの値の大きさの関係は,Fig.\ref{fig:camrad_flight_fzhub}に示す$F_{z\mathrm{hub}}$の4/rev成分の関係と一致することに留意されたい.

    \begin{figure}[htbp]
    \centering
        % 1行目
        \begin{subfigure}[b]{0.45\linewidth}
            \centering
            \includegraphics[width=\linewidth]{photo/psi_alpha_a.png}
            %\caption{$a$}
            \label{psi_alpha_a}
        \end{subfigure}
        \hfill
        \begin{subfigure}[b]{0.45\linewidth}
            \centering
            \includegraphics[width=\linewidth]{photo/psi_alpha_b.png}
            %\caption{$b$}
            \label{psi_alpha_b}
        \end{subfigure}

        \vspace{5mm}

        % 2行目
        \begin{subfigure}[b]{0.45\linewidth}
            \centering
            \includegraphics[width=\linewidth]{photo/psi_alpha_c.png}
            \caption{$c$}
            %\label{psi_alpha_c}
        \end{subfigure}
        \hfill
        \begin{subfigure}[b]{0.45\linewidth}
            \centering
            \includegraphics[width=\linewidth]{photo/psi_alpha_d.png}
            %\caption{$d$}
            \label{psi_alpha_d}
        \end{subfigure}
    \caption{方位角$\Psi$に対する有効迎角$\alpha$の変化}
    \label{fig:psi_alpha}
    \end{figure}

    \begin{figure}[htbp]
    \centering
        % 1行目
        \begin{subfigure}[b]{0.45\linewidth}
            \centering
            \includegraphics[width=\linewidth]{photo/alpha_cl_a.png}
            %\caption{$a$}
            \label{alpha_cl_a}
        \end{subfigure}
        \hfill
        \begin{subfigure}[b]{0.45\linewidth}
            \centering
            \includegraphics[width=\linewidth]{photo/alpha_cl_b.png}
            %\caption{$b$}
            \label{alpha_cl_b}
        \end{subfigure}

        \vspace{5mm}

        % 2行目
        \begin{subfigure}[b]{0.45\linewidth}
            \centering
            \includegraphics[width=\linewidth]{photo/alpha_cl_c.png}
            %\caption{$c$}
            \label{alpha_cl_c}
        \end{subfigure}
        \hfill
        \begin{subfigure}[b]{0.45\linewidth}
            \centering
            \includegraphics[width=\linewidth]{photo/alpha_cl_d.png}
            %\caption{$d$}
            \label{alpha_cl_d}
        \end{subfigure}
    \caption{有効迎角$\alpha$と揚力係数$C_l$の関係}
    \label{fig:alpha_cl}
    \end{figure}

    \par
    Fig.\ref{fig:vi}は誘導速度$v_i$の分布を表している.
    case $a$とcase $d$においては,$\Psi=300°$付近の翼端で$v_i$が増加している.
    Table.\ref{table:attitude_angle}に各ケースにおける航空機の姿勢角を示す.
    case $d$ではピッチ角が0に近いため,ブレードと後流の干渉効果が増大し,Fig.\ref{fig:camrad_flight_fzhub}の
    $F_{z\mathrm{hub}}$における大きな4/rev成分を引き起こしている.

    \begin{figure}[htbp]
    \centering
        % 1行目
        \begin{subfigure}[b]{0.45\linewidth}
            \centering
            \includegraphics[width=\linewidth]{photo/vi_a.png}
            %\caption{$a$}
            \label{vi_a}
        \end{subfigure}
        \hfill
        \begin{subfigure}[b]{0.45\linewidth}
            \centering
            \includegraphics[width=\linewidth]{photo/vi_b.png}
            %\caption{$b$}
            \label{vi_b}
        \end{subfigure}

        \vspace{5mm}

        % 2行目
        \begin{subfigure}[b]{0.45\linewidth}
            \centering
            \includegraphics[width=\linewidth]{photo/vi_c.png}
            %\caption{$c$}
            \label{vi_c}
        \end{subfigure}
        \hfill
        \begin{subfigure}[b]{0.45\linewidth}
            \centering
            \includegraphics[width=\linewidth]{photo/vi_d.png}
            %\caption{$d$}
            \label{vi_d}
        \end{subfigure}
    \caption{誘導速度$v_i$の分布}
    \label{fig:vi}
    \end{figure}



    \begin{table}[H]
        \centering
        \caption{機体の姿勢角}
        \label{table:attitude_angle}
        \setlength{\tabcolsep}{5pt}
        \begin{tabular}{|l|l|l|l|l|}
            \hline                           &    $a$       &    $b$       &     $c$    &      $d$   \\
            \hline ロール角$\Phi$ [deg]       &   3.793      &    3.235     &  2.604     &     4.714  \\
            \hline ピッチ角$\Theta$[deg]      &   -6.533     &    -6.653    &  -6.597    &     -5.748 \\
            \hline 
        \end{tabular}
    \end{table}


    \clearpage

    \section{$r/R= 0.855$における$\alpha$,$C_l$,$U^2$の周波数成分の分析}
    \label{sec:alpha_u2_fft}

    \par
    翼素理論から,ロータハブに作用する荷重$F_{z\mathrm{hub}}$はEq.\eqref{Fzhub}のように表すことができる.
    ここで,シャフト系から見たフラップ角$\beta_0$,$\beta_c$,$\beta_s$は小さいとしている.

    \begin{equation}
        F_{z\mathrm{hub}} = \frac{1}{2}\rho S U^2 C_L  \label{Fzhub}
    \end{equation}

    $U$および$C_L$は,翼のスパン方向における代表(平均)値を示す.
    Eq.\eqref{Fzhub}の右辺において,$U^2$および$C_L$は方位角に依存し,周波数成分を有する.
    Table.\ref{table:product_combination}に列挙されているように,周波数成分の組み合わせにより4/revが生じる.
    この4/rev成分は,これら5つのパターンを足し合わせることで得られる.

    \begin{table}[H]
        \centering
        \caption{4/revとなる周波数成分の組み合わせ}
        \label{table:product_combination}
        \setlength{\tabcolsep}{5pt}
        \begin{tabular}{|l|l|l|}
            \hline product  &    $\alpha$,$C_l$の周波数成分  &    $U^2$の周波数成分     \\
            \hline 1        &       0/rev                   &         4/rev           \\
            \hline 2        &       1/rev                   &         3/rev           \\
            \hline 3        &       2/rev                   &         2/rev           \\
            \hline 4        &       3/rev                   &         1/rev           \\
            \hline 5        &       4/rev                   &         0/rev           \\
            \hline 
        \end{tabular}
    \end{table}    
    
    
    $\alpha$,$C_l$,$U^2$を以下のように方位角$\Psi$に関するFourier級数で展開する.
    ここで,添え字は各周波数成分を表す.

    \begin{align}
    \alpha &= \alpha_0 + \sum_{i=1}^{4} \alpha_i \cos(i\Psi + \delta_i)  \label{eq:alpha_def} \\
    C_l &= C_{l0} + \sum_{i=1}^{4} C_{li} \cos(i\Psi + \varepsilon_i)  \label{eq:cl_def} \\
    U^2 &= U_0^2 + \sum_{i=1}^{4} U_i^2 \cos(i\Psi + \xi_i)  \label{eq:u2_def}
    \end{align}

    $\alpha U^2$と$C_l U^2$の4/revの項$(\alpha U^2)_4$,$(C_l U^2)_4$は以下のように表される.


    \begin{align}
    (\alpha U^2)_4
    &= \alpha_0 U_4^2 \cos(4\Psi + \xi_4)
    + \frac{\alpha_1 U_3^2}{2} \cos(4\Psi + \delta_1 + \xi_3)
    + \frac{\alpha_2 U_2^2}{2} \cos(4\Psi + \delta_2 + \xi_2) \notag \\
    &\quad
    + \frac{\alpha_3 U_1^2}{2} \cos(4\Psi + \delta_3 + \xi_1)
    + \alpha_4 U_0^2 \cos(4\Psi + \delta_4),
    \end{align}
 

    \begin{align}
    (C_l U^2)_4
    &= C_{l0} U_4^2 \cos(4\Psi + \xi_4)
    + \frac{C_{l1} U_3^2}{2} \cos(4\Psi + \varepsilon_1 + \xi_3)
    + \frac{C_{l2} U_2^2}{2} \cos(4\Psi + \varepsilon_2 + \xi_2) \notag \\
    &\quad
    + \frac{C_{l3} U_1^2}{2} \cos(4\Psi + \varepsilon_3 + \xi_1)
    + C_{l4} U_0^2 \cos(4\Psi + \varepsilon_4).
    \end{align}

    \par
    Fig.\ref{fig:product1}~Fig.\ref{fig:product5}に$r/R= 0.855$の$\alpha U^2$,$C_l U^2$の振幅の値を示す.
    $U^2$は,ブレードの運動と前進速度によって決まる翼素への流入速度と,翼素における誘導速度によって決定される.
    $\alpha$は,$U^2$とコレクティブピッチ角及びサイクリックピッチ角によって決定される.
    $C_l$は前述の通り,$\alpha$と翼素の揚力に対する非定常効果によって決まる.

    $\alpha U^2$と$C_l U^2$について,case $b$の値とcase $c$の値の関係は,いかなる組み合わせにおいてもFig.\ref{fig:camrad_flight_fzhub}の結果に対応しない.
    しかしながら,Fig.\ref{fig:product2},Fig.\ref{fig:product3},Fig.\ref{fig:product5}に示されるようにFig.\ref{fig:camrad_flight_fzhub}に見られた
    case $b$,case $c$<case $a$<case $d$という傾向に一致している.
    Fig.\ref{fig:vi}に示されるように,$v_i$は$\Psi = 300°$付近で増加しており,翼端付近ではより顕著な増加が観察された.
    翼端付近での誘導速度の増加がFig.\ref{fig:phi}に見られる流入角の減少を引き起こしている.
    これが結果としてFig.\ref{fig:psi_alpha}の有効迎角$\alpha$の大幅な減少を招いている.
    ここでの$r/R= 0.855$の点に注目した解析では,この$\alpha$の大幅な減少をとらえていないことも,Fig.\ref{fig:camrad_flight_fzhub}と
    Fig.\ref{fig:product1}~Fig.\ref{fig:product5}の差が生じる原因である.
    Fig.\ref{fig:product1}~Fig.\ref{fig:product5}のどの組み合わせにおいても$\alpha U^2$の振幅と$C_l U^2$の振幅の
    各case間での大小関係が一致している.
    これはあらゆるcase間の$\alpha$の周波数成分の大小関係が$C_l$のそれと同じになることを意味している.
    Fig.\ref{fig:alpha_cl}から$C_l$はおよそ$0.1\alpha$と推定される.
    ただし,Fig.\ref{fig:product5}に示されるcase $b$とcase $c$については,揚力傾斜の比率(0.1[1/deg])との差は大きく,
    $C_l$の4/rev成分には大きな翼の非定常効果が見られる.
    Fig.\ref{fig:product1}~Fig.\ref{fig:product5}での$\alpha U^2$,$C_l U^2$の振幅の値から,
    機体の4/rev振動は$\alpha$,$C_l$の4/rev成分と$U^2$の定常成分の積が主成分であることが分かる.
    Ref.\cite{Yoshizaki2023b}では$C_l$の2/revと$U^2$の2/revの振幅に注目し,重心移動により低振動化を議論している.
    精度の高い低周波数の現象で,4/rev振動を説明する方針のためである.
    Fig.\ref{fig:product3}における値の大小関係は,case $b$,case $c$<case $a$<case $d$というFig.\ref{fig:camrad_flight_fzhub}の
    $F_{z\mathrm{hub}}$の傾向と一致している.
    Ref.\cite{Yoshizaki2023b}での$C_l$の2/revと$U^2$の2/revの振幅に注目した分析は,4/rev振動の分析として大きなエラーを持たないと推測できる.
    さらに,Fig.\ref{fig:product3}における各caseの値の比は,他のproductでの比に比べてFig.\ref{fig:camrad_flight_fzhub}での比に近く,
    上記方針(精度の高い低周波数の現象で,4/rev振動を説明する)の妥当性と言える.

    \par 
    Fig.\ref{fig:product_all}に$\alpha U^2$と$C_l U^2$の4/rev成分の振幅を示す.
    Fig.\ref{fig:product_all}での4case振幅の大小関係はFig.\ref{fig:product2},Fig.\ref{fig:product3},Fig.\ref{fig:product5}
    での関係と同一である.
    Fig.\ref{fig:product_all}で示した振幅の値は4/rev振動の主成分であるFig.\ref{fig:product5}に示した振幅より小さい.
    これは,Fig.\ref{fig:product5}($\alpha$,$C_l$の4/revと$U^2$の定常成分)とFig.\ref{fig:product5}
    ($\alpha$,$C_l$の3/revと$U^2$の1/rev)振動の位相が大きく異なり,両振動が打ち消し合うためである.
    また,Fig.\ref{fig:product_all}に示したcase間の値の比は,Fig.\ref{fig:product2},Fig.\ref{fig:product3},Fig.\ref{fig:product5}での
    比よりもFig.\ref{fig:camrad_flight_fzhub}での比に近いとは言えず,スパン方向の1点に注目した,本研究での分析方法の限界と考えられる.

    \begin{figure}[H]
      \centering
      \includegraphics[keepaspectratio, width=0.5\linewidth]{photo/product1.png}
      \caption{$\alpha U^2$,$C_l U^2$の振幅(product 1)}
      \label{fig:product1}
    \end{figure}

    \begin{figure}[H]
      \centering
      \includegraphics[keepaspectratio, width=0.5\linewidth]{photo/product2.png}
      \caption{$\alpha U^2$,$C_l U^2$の振幅(product 2)}
      \label{fig:product2}
    \end{figure}

    \begin{figure}[H]
      \centering
      \includegraphics[keepaspectratio, width=0.5\linewidth]{photo/product3.png}
      \caption{$\alpha U^2$,$C_l U^2$の振幅(product 3)}
      \label{fig:product3}
    \end{figure}

    \begin{figure}[H]
      \centering
      \includegraphics[keepaspectratio, width=0.5\linewidth]{photo/product4.png}
      \caption{$\alpha U^2$,$C_l U^2$の振幅(product 4)}
      \label{fig:product4}
    \end{figure}

    \begin{figure}[H]
      \centering
      \includegraphics[keepaspectratio, width=0.5\linewidth]{photo/product5.png}
      \caption{$\alpha U^2$,$C_l U^2$の振幅(product 5)}
      \label{fig:product5}
    \end{figure}

    \begin{figure}[H]
      \centering
      \includegraphics[keepaspectratio, width=0.5\linewidth]{photo/product_all.png}
      \caption{$\alpha U^2$,$C_l U^2$の振幅}
      \label{fig:product_all}
    \end{figure}


    \begin{figure}[htbp]
    \centering
        % 1行目
        \begin{subfigure}[b]{0.45\linewidth}
            \centering
            \includegraphics[width=\linewidth]{photo/phi_a.png}
            %\caption{$a$}
            \label{phi_a}
        \end{subfigure}
        \hfill
        \begin{subfigure}[b]{0.45\linewidth}
            \centering
            \includegraphics[width=\linewidth]{photo/phi_b.png}
            %\caption{$b$}
            \label{phi_b}
        \end{subfigure}

        \vspace{5mm}

        % 2行目
        \begin{subfigure}[b]{0.45\linewidth}
            \centering
            \includegraphics[width=\linewidth]{photo/phi_c.png}
            %\caption{$c$}
            \label{phi_c}
        \end{subfigure}
        \hfill
        \begin{subfigure}[b]{0.45\linewidth}
            \centering
            \includegraphics[width=\linewidth]{photo/phi_d.png}
            %\caption{$d$}
            \label{phi_d}
        \end{subfigure}
    \caption{流入角$\phi$の分布}
    \label{fig:phi}
    \end{figure}




